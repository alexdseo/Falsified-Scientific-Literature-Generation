
\documentclass{article}
\usepackage[utf8]{inputenc}
\usepackage{authblk}
\usepackage{textalpha}
\usepackage{amsmath}
\usepackage{amssymb}
\usepackage{newunicodechar}
\newunicodechar{≤}{\ensuremath{\leq}}
\newunicodechar{≥}{\ensuremath{\geq}}
\usepackage{graphicx}
\graphicspath{{../images/generated_images/}}
\usepackage[font=small,labelfont=bf]{caption}

\title{A new Gene Fusion assay looks at how the physical}
\author{Christian Wright\textsuperscript{1},  Kelly Brown,  Stephanie Henson,  Colin Ramsey,  Richard Brooks,  Stanley Knight,  Emma Osborne,  April Lopez,  Thomas Dyer}
\affil{\textsuperscript{1}Peking University School of Oncology}
\date{August 2012}

\begin{document}

\maketitle

\begin{center}
\begin{minipage}{0.75\linewidth}
\includegraphics[width=\textwidth]{samples_16_178.png}
\captionof{figure}{a woman in a red shirt and a red tie}
\end{minipage}
\end{center}

A new Gene Fusion assay looks at how the physical and chemical makeup of one of the secret pre-RNAs that form parts of normal human beings in the laboratory. The results were obtained by the Japanese National Science Foundation (JNSF) based on a combination of accelerator studies the cores of a plant\'s intestines and of plants\' intestines.

Gene Fusion was an exciting step forward in biologist research using methods based on real animals or animals that model an organism and replace traditional methods with viruses for prediction of its outcomes.

The findings may offer a further mechanism for identifying new biological targets for the next generation of animals to develop synthetic bio-made human cells.

The ancestral root systems of domesticated animals in Japan looked like a fully grown normal human one, but one of them looked like a rare, genetically altered animal developed with immune disease in it.

This sequence of DNA, which means that the genes must have the power to re-manufacture the human ancestor\'s DNA, was drawn from microorganisms grown in the lab and used by Japan\'s JNSF research lab.

To use the Gene Fusion assay, JNSF developed material from microorganisms that are no longer alive and those grown in Japan. It also applied a method known as isolated transcriptase recombination, or by G+P3C, to the assay to add yet another human (genetic variant) or animal (D3D) to the sample at risk of misalignment.

"This assay allows people to compare the genetic variants their ancestors produced in plant and animal DNA with one from a human to match similar variants identified in their DNA," said Xiang Sheng, research lead author and a neurobiologist at JNSF\'s animal stem cell and cellular research laboratory in Tokyo.

Fusion results represent the first study of a novel assay containing potential to use biologically modified materials.

The DNA sequence for Gene Fusion is called "Neoplastic Nodes." In this phase of the study, the enzyme produced by a plant called the usual progenitor cell is modified to synthesize the liver hormones needed to make protons, and has the ability to synthesize between 150 and 250. This process produces large, single nucleotide polymorphisms, which indicates that all of the human genes and the genetic material in a family member might appear as with normal genes.

There is no treatment for Neoplastic Nodes. For one thing, the gene involved has only the most basic conditions in its origins to use as a biological agent that can change its structure. Another requirement is that it will produce damage that will kill the cells if detected.

Following the study, JNSF granted Japan access to the Gene Fusion assay for the first time in the next step in its genetic research project, proposed a genus for different proteins that may link to, isolate and recombine with the group of cells responsible for the genetic process.

Source: Hui Li, Jinglan Wang, Gil Mor, Jeffrey Sklar

Japan\'s Mathematical Arm Declined in Photographs


\end{document}