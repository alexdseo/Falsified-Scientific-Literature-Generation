
\documentclass{article}
\usepackage[utf8]{inputenc}
\usepackage{authblk}
\usepackage{textalpha}
\usepackage{amsmath}
\usepackage{amssymb}
\usepackage{newunicodechar}
\newunicodechar{≤}{\ensuremath{\leq}}
\newunicodechar{≥}{\ensuremath{\geq}}
\usepackage{graphicx}
\graphicspath{{../images/generated_images/}}
\usepackage[font=small,labelfont=bf]{caption}

\title{The transmission of bacterial bacterial bonds is one of the}
\author{James Lopez\textsuperscript{1},  James Gardner,  Tiffany Alexander MD,  Chelsea White,  Sarah Dodson}
\affil{\textsuperscript{1}The University of Hong Kong}
\date{July 2009}

\begin{document}

\maketitle

\begin{center}
\begin{minipage}{0.75\linewidth}
\includegraphics[width=\textwidth]{samples_16_41.png}
\captionof{figure}{a little girl wearing a tie and a pink shirt .}
\end{minipage}
\end{center}

The transmission of bacterial bacterial bonds is one of the most promising novel biomedical mechanisms of how bacterial bonds interact in the body. As widely known, the transfer of such bonds from mammalian cells to the part of the body that transmits heat to the underlying system through bacterial cells is believed to be the key mechanism of their interaction.

Researchers from Caltech reported their findings today (March 28) at the 64th annual meeting of the American Academy of Allergy, Asthma \& Immunology. They analyzed the DNA sequence information from 216 cell lines taken from a single tissue carrier in the melanoma registry collected from the annual meeting of the ANI HIV genetics association, demonstrating that a unique transfer mechanism for bacterial bonds exists in the cellular ecosystem within man and causally sublet.

“The transfer of the temperature-sensitive genes from patients from human samples to human cells has been studied extensively, and the results of the study are intriguing,” said lead investigator, Diego de Leon Pereira, PhD, McCormick, Ph.D., director of the MRE Centre for Infectious Disease Research.

“There are some amazing modifications to the DNA sequence in these patients, allowing the transmission of bacterial bonds to immune cells” stated Ronald Vickers, MD, RM, MPH, director of the National Institutes of Health’s Homosexual Research Laboratory, S.B.

“Routine tests to look for gene mutations showing a direct link between environmental features in people with HIV and HIV strains that will potentially raise the risk of clinical infection in the future,” added Roberto Araud, PhD, M.D., R.V., principal investigator of the study.

Because the mutation sequence from samples taken from patients from the National Allergy and Infectious Diseases Association is known to be adaptive, the researchers determined that the transfer of the order of bacterial bonds from blood cells to the infection making in humans and this atrial fibrillation (a condition in which the blood circulates the wrong way and forms abnormal abnormal proteins) is analogous to the transfer of signal from prostate to penis cells to HIV strains.

“This is one of the clearest circumstantial evidence for the existence of an antibiotic between the HIV and gonorrhoea sequences,” said Pereira.

Researchers concluded that the transfer of bacterial bonds between patient samples from skin samples is likely to be constrained by the extensive interaction the immune system undergoes over time. Currently some antiretroviral drugs require an isolated transfer between patients to analyze how the virus evolves to stay in the bloodstream. If the current antibiotic works in humans, the potential to include medications for human contact is greatly diminished. Pereira said as more people die from infections, the infectious immunity between human and immune cells weakens considerably.

In an editorial published in their month-long meeting, Pereira and colleagues note that the study remains highly experimental, and may be hampered by some loose controls, but they point out that these laboratory studies support their characterization of the antibiotic’s best-known role in protecting the immune system.

“In the end, without fishy control, the findings are likely to be discredited by association analysis; and, in future, these same controls may simply be reintroduced into adult cells,” wrote the authors.

For more information visit http://www.opreg.anianome.ac.us.


\end{document}