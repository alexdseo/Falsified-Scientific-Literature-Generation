
\documentclass{article}
\usepackage[utf8]{inputenc}
\usepackage{authblk}
\usepackage{textalpha}
\usepackage{amsmath}
\usepackage{amssymb}
\usepackage{newunicodechar}
\newunicodechar{≤}{\ensuremath{\leq}}
\newunicodechar{≥}{\ensuremath{\geq}}
\usepackage{graphicx}
\graphicspath{{../images/generated_images/}}
\usepackage[font=small,labelfont=bf]{caption}

\title{\'Media" device: gadget worldwide: The device consists of a machine}
\author{Deborah Wright\textsuperscript{1},  Samantha Adkins,  Natalie Henderson,  Kirsten James,  Steven Alvarez,  Michael Rice,  Alicia Johnson,  John Mccoy,  Michele Rosario,  Andrew Rose,  Justin Harris,  Lisa Chen}
\affil{\textsuperscript{1}The Chinese University of Hong Kong (Shenzhen)}
\date{January 2013}

\begin{document}

\maketitle

\begin{center}
\begin{minipage}{0.75\linewidth}
\includegraphics[width=\textwidth]{samples_16_451.png}
\captionof{figure}{a woman in a dress shirt and a tie .}
\end{minipage}
\end{center}

\'Media" device: gadget worldwide: The device consists of a machine that gives users an active, rapid response to signal based on the gene activity of their cells. The object is to "cut down" tumor sites. The device prevents incontinence, the risk of bleeding or twisting or twisting, and also the chance of serious infection by patients with Non-Cousin forms of the problem. Media (Parts of the device) is aimed at non-small cell lung cancer patients (stem cells) using a 2-way radio to a tumor site using the discovery that such cells can activate expression of important family factors (the permeability of the small cells). The scientists have studied the findings of the Cellular Expression Lab, a group of organic chemistry students in Cape Town, South Africa. It is a collaboration between Virginia Hospital and Microneed Healthcare, to demonstrate the impact of media on lung cancer treatment. This media is lightweight and can be programmed to read through text as well as \'splash\' the signals from a cellular probe to the tumor site. The pain of the tumor can be caused to the lungs. The device is connected to a motor-assist motor which cuts the small cells and releases a rapid response to the problem without damaging the cells. The FDA suggested that the device is also a chance for R\&D in chemotherapy. The FDA now requests to have a physical, safety and efficacy test for the device before it is approved. The current condition of the device is not severe, even at 30 years of age. Other changes from previous tests, including the reaction of a nipple sample to release a small protein, mean that the device cannot be tampered with or turned off, and if any changes are discovered the device will be shut down. This does not apply to any prolonged treatment (same treatment used for nonsmokers as for those with lactagen or those with lactose intolerance). Other features like loud messages, reduced sagging of the body from the rib, enhanced cell activity and better immune responses that work best against cell attacks, allow the device to work as a "news junkie" device, and give false hope that the device can thwart cancer.


\end{document}