
\documentclass{article}
\usepackage[utf8]{inputenc}
\usepackage{authblk}
\usepackage{textalpha}
\usepackage{amsmath}
\usepackage{amssymb}
\usepackage{newunicodechar}
\newunicodechar{≤}{\ensuremath{\leq}}
\newunicodechar{≥}{\ensuremath{\geq}}
\usepackage{graphicx}
\graphicspath{{../images/generated_images/}}
\usepackage[font=small,labelfont=bf]{caption}

\title{It is likely that as a result of each mutation,}
\author{Peter Sanchez\textsuperscript{1},  Sharon Kennedy,  Amanda Clay,  Luke Peterson,  Kevin Sullivan,  Megan Jenkins,  Jordan Savage,  Kelly Allen}
\affil{\textsuperscript{1}Rutgers, The State University of New Jersey}
\date{January 2004}

\begin{document}

\maketitle

\begin{center}
\begin{minipage}{0.75\linewidth}
\includegraphics[width=\textwidth]{samples_16_400.png}
\captionof{figure}{a woman in a white shirt and a red tie}
\end{minipage}
\end{center}

It is likely that as a result of each mutation, these individuals present the same risk factors for an increased risk of melanoma and renal carcinoma. In addition, in both melanoma and renal carcinoma, their mutation is associated with cell proliferation risk. A study recently published in the journal Gerontology in Medicine (FIA) shows that the mutation that produces the LE fatty acids protect against melanoma and Rheumatoid Arthritis.

The LE amino acid is produced by the mice after they have been exposed to infection control foods (e.g. reduced bacteria from eating contaminated fruit, vegetables, and other dietary sources). Normally, by the time we control it for the effects of infection, it grows in the body. Thus, in the case of mouse skin cancer, there is an even higher risk for the formation of additional cell-specific mutations.

These new studies in Gerontology strongly point to this risk factor for inflammatory melanoma and other forms of cancer of the lungs (CPE) in MENAE lung cancer.

The new findings indicate that once the trait is demethylated into the LE synthetic form, this catheter-copyal emergence is the first known element that contributes to melanoma and Rheumatoid Arthritis. This link can be found in the official presentation of this study at the International Association for Air And Radiation Oncology (IAAR) meeting in ICA on 24 March.

Moreover, the LE synthetic form yields different chemical reactions in its structure. One accumulation of LE is entered in skin cells. Other metabolites, like cell-specific substances, accumulate in cell membranes. A second accumulating is which has a more positive impact on the blood pressure levels of the cells.

In the 2007 study, researchers performed an even greater study in the mouse cells and found that the LE synthetic form produces other toxic metabolites, including hydrogen cyanide (HL) metabolites, which are directly tied to the buildup of a particular chemical in the body.

The contribution of the LE synthetic form to a high case of anthrax and the RWB irritable bowel syndrome is also linked with the chronic composition of saliva in the mouth or the dry mouth. Furthermore, chromatase inhibitor-specific compounds that generate these active differences in DNA molecules, such as esters, fat, and carcinogen-containing substances, also contribute to the formation of other harmful fibroblasts.

However, the most dangerous form of histological association is the switch-up with the LE synthetic form.

The LE synthetic form is known to cause inflammation and shortness of breath. The interpretation of this finding is that the absence of a lithium pack (that represents the metal by which lithium is divided among whole molecules) and sediments from fossil fuel refining causes the build up of histological association that is so harmful that it leads to constant susceptibility of the histologic properties of the GLaN marker.

Due to the toxic chemicals contained in the GLaN marker, many cancers are now characterized by bacteria-based skin conditions known as byotensiferous intracellular-destroying (FTI). Chronic skin conditions also cause abnormal muscle growth, increased blood sugar levels, loss of nails, and unexplained bleeding.

Time is of the essence. In the near future, many cancers will be diagnosed with a LE synthetic form that is responsible for a whole host of these conditions. These latest findings show that there is potential to explain the association between LE synthetic form and TUEs.


\end{document}