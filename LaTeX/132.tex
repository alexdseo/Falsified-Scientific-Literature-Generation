
\documentclass{article}
\usepackage[utf8]{inputenc}
\usepackage{authblk}
\usepackage{textalpha}
\usepackage{amsmath}
\usepackage{amssymb}
\usepackage{newunicodechar}
\newunicodechar{≤}{\ensuremath{\leq}}
\newunicodechar{≥}{\ensuremath{\geq}}
\usepackage{graphicx}
\graphicspath{{../images/generated_images/}}
\usepackage[font=small,labelfont=bf]{caption}

\title{The body is awash with hope. Now it can soar}
\author{Charles Cox\textsuperscript{1},  Erika Ferguson,  Kathryn Krause,  Michael Phillips,  Priscilla Harper,  Benjamin Richardson}
\affil{\textsuperscript{1}University Hospital Erlangen}
\date{April 2014}

\begin{document}

\maketitle

\begin{center}
\begin{minipage}{0.75\linewidth}
\includegraphics[width=\textwidth]{samples_16_132.png}
\captionof{figure}{a woman in a white shirt and a red tie}
\end{minipage}
\end{center}

The body is awash with hope. Now it can soar into the stratosphere. Something deep and mysterious down there must be back on its feet. Or, perhaps, the cells are in danger of dying.

It’s part of a new chronic disease called “malignant melanoma,” with the apparent narrative that this is the future — and now the past is our guide.

It’s something every tumor undergoes, and what this tells us is that the tumor cells are clinging to the hope of the future and they’re all growing forward at about the same rate as their predecessors. The same thing happens in cancerous cells.

This, it turns out, is often called apoptosis. And, yes, it’s important.

Especially in breast cancer where, at the worst time of the day or night, cancer cells are alive and well. It’s where the cells are happy, strong and alive, and at that point they realize their life is diminishing, the cancer cell can become more efficient at killing it, and the remaining cells have to rely on treatments and gut microbes.

This case showed that in breast cancer, apoptosis can lead to cell failure and, unfortunately, bad prognosis. Too bad, too, for many of us, as we discover how to destroy cancer cells without repeling their growth and bone density in our bodies. It turns out there’s only one way to death it! The cells can become stuck. Worse, the cells stop growing and the cells are dying quickly, at times more than we can hope for.

As all clinical trials, however, point toward the future, with antibiotics and biopsies, and perhaps human screening, the lymph cells that once turned tumors into malignant are primed for reversal. They’re not going to get better any faster and, if they do, the immune system will be even more robust.

“The frustrating thing for many scientists in cancer is that there’s no blood testing,” says Cynthia Hoffmann, Ph.D., director of the National Institute of Allergy and Infectious Diseases. “There’s no tissue matching between human breast cancer cells and their cells on the path to recovery.”

It’s a problem that stems from a “lethal” mutation that occurs in over 130,000 breast cancer cells, which as Hoffmann explains, indicates an increase in radical activity, tumor growth, stress and hormonal variations.

So what happens to these cells if they’re destroyed, or die young, and thus seem more susceptible to developing cell injury or death?

A team led by Hoffmann and colleagues in the School of Medicine of the National Cancer Institute put it to a test this week with a novel, blood-based regimen. That was great news for researchers, and they weren’t alone. There were some deaths.

The study was published Monday in the journal Cell Stem Cell. They write, “As the possibilities and biopsies become more robust, these leukemia-fighting combinations are likely to be applied to the industry of cancer treatments.”

Even if two chemotherapies are given on a daily basis by the FDA on for the treatment of blood cancers, the frequency of leukemia-fighting combinations can depend on whether one is from the chemotherapy on or from the skin. To qualify for the treatment, doctors must be recruited from the American Cancer Society, the National Institute of Allergy and Infectious Diseases, the American Life Congress and other cancer research organizations. In addition, the national Institute of Cell Stem Cell Data, a professional organization affiliated with the National Cancer Institute, supports grant applications to conduct clinical studies and clinical trials on lymph cells and leukemia cells that may yield new medical benefits and to develop diagnostic methods for cancer treatment.

And Hoffmann and her team advanced some promising researchers to the American Society of Clinical Oncology. They added 25 more to their trials, putting additional help up front.

The three additional trials carried out by the team will be named, “NLSC has partnered with BPO Therapeutics to advance its drugs to the clinical trials of inflammatory bowel diseases.” To that end, an estimated 3,000 patients have already been selected for the trial. “The rate of survivorship in this setting is significantly higher than in other areas of cancer,” Hoffmann concludes.


\end{document}