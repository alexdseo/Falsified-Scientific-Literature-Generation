
\documentclass{article}
\usepackage[utf8]{inputenc}
\usepackage{authblk}
\usepackage{textalpha}
\usepackage{amsmath}
\usepackage{amssymb}
\usepackage{newunicodechar}
\newunicodechar{≤}{\ensuremath{\leq}}
\newunicodechar{≥}{\ensuremath{\geq}}
\usepackage{graphicx}
\graphicspath{{../images/generated_images/}}
\usepackage[font=small,labelfont=bf]{caption}

\title{Methyltransferase inhibitors Improve the Effect of Chemotherapeutic Agents in SW48}
\author{Brian Davis\textsuperscript{1},  Michelle Woods,  Michael Nguyen,  Joseph Hawkins,  Julie Ware,  Dr. Nicole Wolfe,  Patricia Lee,  Rodney Cunningham,  Mary Hicks}
\affil{\textsuperscript{1}University of Milan}
\date{May 2009}

\begin{document}

\maketitle

\begin{center}
\begin{minipage}{0.75\linewidth}
\includegraphics[width=\textwidth]{samples_16_233.png}
\captionof{figure}{a woman in a white shirt and a red tie}
\end{minipage}
\end{center}

Methyltransferase inhibitors Improve the Effect of Chemotherapeutic Agents in SW48 and HT-29

In the United States, dentigners and other medical professionals utilize chemical conjugates and biofilms to conduct therapeutic drug conjugates used to treat medicinal pain or other specific conditions. Radiotherapy of acid andmethylcyclocephaly are the principal drugs typically used in drug conjugates, and the combination of two secondhand chemotherapeutic agents can significantly increase the effectiveness of these compounds in treating patients with these conditions. These drugs contain related components of certain foods and drugs, which in turn can impact the molecular profile of multiple various biochemical and physiological compounds.

In furthering their research on the use of chemotherapeutic agents in human disease, medical specialists at Southwest University say that different molecular studies indicate that numerous types of molecular agents can alter the composition of narcotics, often leading to dangerous side effects. Without doubt, these molecular agents may have other and contributing properties that may bring new therapeutic treatments forward, especially in oral drugs. While the discovery of certain chemicals that change the composition of medications for anti-cancer agents are excellent, there is currently no evidence that these chemicals have been studied in humans. These molecular combinations may have been present in many different drug class molecules, potentially affecting the molecular structure of the drugs by making these compounds easier to swallow.

The link between the molecular changes in the composition of compounds in the body and the interactions of these substances is revealing, with further studies indicating that while the molecular composition of daily doses of chemotherapeutic agents may change, the levels of the molecular changes that differentiates them are still very limited.

Most of this is due to the level of variation in the molecular structure of drugs. Although many compounds might enhance efficacy, some may not, which means there is more than one drug in the pool that actually improves efficacy. What appears to be the biggest factor in the progression of drug interactions is the total number of interactions and interactions between these substances, with only a small fraction affected by combinations or alterments of compounds.

A study published in 2004 in the Journal of the American Chemical Society led to a number of studies exploring the effect of certain compounds in medicine, with the leading study examined the conduct of chemotherapeutic agents in animal models of addiction and cancer; as well as clinical trials where the interactions resulted in therapeutic effects. The question was, which chemicals were the most significant alterations. The results of the study showed that 4 of 5 toxic properties of therapeutic agents and other metabolites were consistent with the interaction of drugs with various receptors.

In 2015, research is ongoing to examine the interaction between different chemical compounds, with the greatest potential discovery finding being in the combination of two secondhand chemotherapeutic agents. For these directions, current methods of combining existing drug class molecules with other chemical molecules have a long history of following the same path. However, numerous studies with differing methods is currently focused on the relationship between several active and inactive chemicals, with many predicting that the relationships will continue to evolve.

To learn more about the linking between chemical types and interactions, or the study for this article, please contact Ruth-Ann Landers at care@western-releasage.net or email protected For a transcript of this article, please visit: https://www.west-releasage.com/1/09th/02/2016055221407.pdf.

PRESS CONTACT: RuthAnn Landers

SR

SWU College of Nursing

530.925.64093


\end{document}