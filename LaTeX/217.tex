
\documentclass{article}
\usepackage[utf8]{inputenc}
\usepackage{authblk}
\usepackage{textalpha}
\usepackage{amsmath}
\usepackage{amssymb}
\usepackage{newunicodechar}
\newunicodechar{≤}{\ensuremath{\leq}}
\newunicodechar{≥}{\ensuremath{\geq}}
\usepackage{graphicx}
\graphicspath{{../images/generated_images/}}
\usepackage[font=small,labelfont=bf]{caption}

\title{By Rounak Nassirpour

(JAMA) – In mice, inflammation is associated with}
\author{Pamela Flowers\textsuperscript{1},  Aimee Walker,  Kimberly Compton,  Jason Randolph,  Sheila Burke,  Veronica Khan,  Jason Wright,  Jodi Smith,  Dr. Jermaine Gomez,  Denise Collins,  Michelle Smith,  Connie Eaton}
\affil{\textsuperscript{1}Hong Kong Hospital Authority}
\date{February 2012}

\begin{document}

\maketitle

\begin{center}
\begin{minipage}{0.75\linewidth}
\includegraphics[width=\textwidth]{samples_16_3.png}
\captionof{figure}{a woman and a man are posing for a picture .}
\end{minipage}
\end{center}

By Rounak Nassirpour

(JAMA) – In mice, inflammation is associated with mastitis in many more developing human cancers, regardless of the tumor type. In this new study, investigators show that cystidigenesis—a technique for manipulating genetically modified enzymes that serve to enhance the gene expression in the cells of human breast cancer cells—also holds potential for tissue salvage efforts in this tumor. In addition, applying cystidigenesis to tumor-free or virulent breast cancer cells simultaneously opens the possibility of using cystidigenesis to enhance the number of cells in the pathway that support cancer development.

“Our finding that a mutation in a gene activator in mice affected cancer cells in the next stage of development, coupled with powerful interactions with other genes that are key to tumor defense, are one of the most promising trends in breast cancer science,” said Dr. Aleksandrit Novakova, professor of radiology and anatomology and senior author of the study published in the current issue of JAMA. “This research builds on a very promising public-health technology that is a model for the destruction of cancer tissue from the tumor, although how effective and effective the treatment should be remains a mystery.”

Livonia Baskin, a senior author on the study, said that cystidigenesis has had some exciting limitations. “Because it is not used strictly to destroy cancerous tissue, we can say that cystidigenesis alone, and despite its toll on cell function, is so effective in destroying the DNA we create in these tissues in the near future,” she said. “The technique for harvesting and purging blood from colon and rectum cancers is very promising, however. We know that lymphoedema, a serious form of cancer that kills about 70 percent of the patients who get it, involves using a C- cell gate (C-stemo-turbine) cell to fill in gaps in DNA that are then transmitted to tissue with infusions of cystidigenesis. It is not possible to make these non-cancerous C- cells with cystidigenesis.”

The research, led by Karol Weiss, principal investigator at the Center for Radiological Research, Biomedical and Medical Engineering at the University of Arizona, showed that conductive cisplatin was also involved in tissue salvage and as a method of immunosuppression. Wyeth Corp. in Seattle, which manufactures the drug C-talide MabThera, supplied tissues containing cystidigenesis using its products, the researchers said.

“We wanted to demonstrate the utility of these targeted preparations in tumor-free research to improve and accelerate the data-driven, personalized treatment of cancer metastatic disease and relapse,” said Dr. Weiss. “This trial is proof that these targeted preparations are equally effective in non-cancer forms of cancer, although the mechanisms and types of cell offload mutations differ.”

Results showed that the mutated cystidigenesis structure is critical to better understanding how the protein-protein interaction between common cancer cells and cell genes leads to cancer formation and progression of new forms of tumors. The scientists also showed that cystidigenesis affected stem cells in markers that prompt the proliferation of new and previously-harmful proteins in the DNA.

The findings were published in JAMA.

\#\#\#

Reference: Baskin et al. C-Cell Edge (Journal of Radiology, 2002). Cancer progress in breast cancer-free and virulent breast cancer cells. Nature, 220 DOI:10.1038/ JAMA.03353-1217.18

Priest Clinic, U-0121-0702H, Chicago, IL.

Notes:

Stem Cell Technology, disseminated through Interdisciplinary Research Center, Pacific Rim Regional Molecular Biology Institute (PAMRIB), 520 N. Milwaukee Ave., Suite 100, Chicago, IL, 60031.

Contact: Claire Collins, Assistant Director – Administration, the VA Medical Center, 414 W. Washington Blvd., Washington, DC 20024.

About AstraZeneca: AstraZeneca is a multinational pharmaceutical company with leading research-based brands. Its patented drugs are used in hospitals worldwide to treat serious and life-threatening conditions, and to improve the quality of life for patients. In February 2011, AstraZeneca agreed to buy US-based Millennium Pharmaceuticals, Ltd. for \$5.8 billion (€4.8 billion). Millennium has many new products on the market, including a new treatment for Alzheimer’s diseas

\end{document}