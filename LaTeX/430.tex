
\documentclass{article}
\usepackage[utf8]{inputenc}
\usepackage{authblk}
\usepackage{textalpha}
\usepackage{amsmath}
\usepackage{amssymb}
\usepackage{newunicodechar}
\newunicodechar{≤}{\ensuremath{\leq}}
\newunicodechar{≥}{\ensuremath{\geq}}
\usepackage{graphicx}
\graphicspath{{../images/generated_images/}}
\usepackage[font=small,labelfont=bf]{caption}

\title{47

"In the current clinical literature, STIM1 generally inhibits calcification of}
\author{Kevin Orozco\textsuperscript{1},  Brian Jones,  Troy Smith,  Katherine Martinez,  Michael Swanson,  Mike Bridges,  Sara Rowe,  Timothy Lloyd,  Shelia Simpson}
\affil{\textsuperscript{1}Rutgers, The State University of New Jersey}
\date{January 2009}

\begin{document}

\maketitle

\begin{center}
\begin{minipage}{0.75\linewidth}
\includegraphics[width=\textwidth]{samples_16_430.png}
\captionof{figure}{a man in a suit and tie is smiling .}
\end{minipage}
\end{center}

47

"In the current clinical literature, STIM1 generally inhibits calcification of the cancer and defects and conditions, such as differences in key histological pathways and expression of inflammatory cytokines," says Professor Ed Tatum, first author of the study.

UC San Diego researchers are investigating an indirect attack of the STIM1 receptor to inhibit the toxic components of inflammatory disorders in patients with stage IV colorectal cancer.

Detecting this toxic intersection between STIM1 (a preclinical proven suppression regulator) and engineered cell blocking agents (key growth factors) or cellular agents (link-in tools), the investigators discovered that a clinical course of inhibition is necessary for maintaining safe and effective development of STIM1. In most large studies, chemotherapeutic agents (drugs) demonstrated positive effects during Phase III clinical trials. These toxic effects were validated and documented for the first time.

Before commercializing STIM1, researchers would have needed to confirm that the drug suppressed inflammatory IMPLO2 γ genotypes previously implicated in immune T-cell proliferation. These were considered to be a variable risk factor for inflammatory response and had the potential to be cause for resurgence in patients with breast, ovarian, renal and colorectal cancers. Current research has shown that immune related T-cell responses in patients with breast cancer are similarly potent for inflammatory response.

"Taken together, the combined study reports that inhibition of STIM1 does create tumor regrowth with zero differentiation between benign and toxic parameters," says the scientist who recruited patients who were Stage I colorectal cancer patients with advanced treatment of the disease.

In contrast, beneficial agents known as peptide inhibitors are already taken orally at much higher doses. A first-of-its-kind study that studies how antigens and peptides interact during integrative approach to tumor management in patients with advanced breast cancer are published in JAMA on March 28.

Originally published in the Journal of Clinical Oncology


\end{document}