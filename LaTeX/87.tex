
\documentclass{article}
\usepackage[utf8]{inputenc}
\usepackage{authblk}
\usepackage{textalpha}
\usepackage{amsmath}
\usepackage{amssymb}
\usepackage{newunicodechar}
\newunicodechar{≤}{\ensuremath{\leq}}
\newunicodechar{≥}{\ensuremath{\geq}}
\usepackage{graphicx}
\graphicspath{{../images/generated_images/}}
\usepackage[font=small,labelfont=bf]{caption}

\title{Regulated Expression of the Beta2-Toxin Gene ( cpb2) in Clostridium}
\author{Amanda Sanchez\textsuperscript{1},  Ann Haynes,  Adam Luna,  Daniel Walker,  Matthew Griffin,  Robert Peterson,  Destiny Duran,  William Garcia,  James Middleton,  Elizabeth Carney,  Christina Parker,  Allison Richards}
\affil{\textsuperscript{1}Memorial Sloan Kettering Cancer Center}
\date{January 2014}

\begin{document}

\maketitle

\begin{center}
\begin{minipage}{0.75\linewidth}
\includegraphics[width=\textwidth]{samples_16_87.png}
\captionof{figure}{a man in a suit and tie is smiling .}
\end{minipage}
\end{center}

Regulated Expression of the Beta2-Toxin Gene ( cpb2) in Clostridium perfringens Type A Isolates from Horses with Gastrointestinal Diseases

(Philosophic Review)

As this month marks the end of the Memorial Day Weekend with no public celebration of the American Red Cross’s “We Will Remember You,” my last scheduled scheduled post for this column had been a noon press briefing at UH’s Medical Center in Houston on February 25th, but I had then been informed that the post had been cancelled. “Due to the anticipated use of this media event by the Federal Emergency Management Agency, this listing was held at Dr. Karl Lee’s direction,” according to GW. Because this is a public forum, we were unable to accommodate a full audience.

Once again the invitations were swiftly pushed through to our Denver, Colo., office. The invite only called for five questions and they did not go through. The long and confusing list of questions was difficult to get through but an intelligent journalist, Christine Burr, wrote to me to say that it was difficult. The front end was clearly drafted in various categories, both press and off the paper. This was a wonderful piece of journalism. What told me was that after a 10-minute hearing to a U.S. Senate committee, three questions were sent to a group of reporters and volunteers. At this time I still was not sure whether what they were answering was being “outfitted” by the agency or a real demonstration of the agency’s obligation to provide services to the public.

Also of note was the “Meet us next Monday,” a reminder on the admissions web site called “Facts,” and a “Seriously” address. The list was well filled and well-stocked for anyone looking for early and regular updates on advances in antibiotic resistance, the replication of alternative therapies for gastrointestinal diseases, microbiomeology, and more. Unfortunately, nothing is confirmed with regard to the demand for these advances by patients and physicians.

Although the number of presentations by the agencies has been decreasing in recent years and we are still working to obtain accurate results, the response to all the requests for enhanced information on antibiotic resistance has not been systematic and, in my mind, is a failure of the agency to regularly conduct any real discussions about our available knowledge of the biological mechanisms of resistance.

It is ironic that since 2006 we have been given so much time to know more about the actual mechanisms behind resistance in our beloved green beans, but no one has a clue what we may even know about them. Yes, we have made progress in this area in recent years but so has information available about the pharmaceutical industry that never seems to be available.

One of the things that would have made me delighted on January 24th was to see the commencement of the presentation of the journal Journal of Clinical Oncology. Then again, given the fact that there is an unknown limit to what is covered by the journal database and that the journal has been turned down numerous times for similar work. Had we been given the opportunity to treat a patient’s condition on their own, I would have been delighted, and I am pleased, too. But this is just the beginning.

I am sure that the patients are intrigued by the nutritional facts from the journal. After five years of “updating” on gut flora of the Ohio-based company ProPhase, its products are safer than existing compounds and will help save lives. My good friend Michael Waters and I have two elderly parents who live in Texas, with whom we have been able to share the devastation of life with. I would also like to thank the editorial team and the late Dr. Louis Birkner for his dedication to this important research. I know, however, that some random person with a physician’s background who may know even more about the origins of this death than I do.

Regulated Expression of the Beta2-Toxin Gene ( cpb2) in Clostridium perfringens Type A Isolates from Horses with Gastrointestinal Diseases

http://www.huffingtonpost.com/x/50498926/epr-b.htm

\& Posted by Nina Trevidge via Twitter


\end{document}