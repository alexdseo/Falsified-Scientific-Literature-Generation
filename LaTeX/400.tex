
\documentclass{article}
\usepackage[utf8]{inputenc}
\usepackage{authblk}
\usepackage{textalpha}
\usepackage{amsmath}
\usepackage{amssymb}
\usepackage{newunicodechar}
\newunicodechar{≤}{\ensuremath{\leq}}
\newunicodechar{≥}{\ensuremath{\geq}}
\usepackage{graphicx}
\graphicspath{{../images/generated_images/}}
\usepackage[font=small,labelfont=bf]{caption}

\title{By Elizabeth Halut

You may know the MITF germline mutation, which}
\author{Teresa Sullivan\textsuperscript{1},  Charles Olson,  Robert Smith,  Jimmy Swanson}
\affil{\textsuperscript{1}King Faisal University}
\date{April 1999}

\begin{document}

\maketitle

\begin{center}
\begin{minipage}{0.75\linewidth}
\includegraphics[width=\textwidth]{samples_16_186.png}
\captionof{figure}{a woman and a man are posing for a picture .}
\end{minipage}
\end{center}

By Elizabeth Halut

You may know the MITF germline mutation, which is contained in two genes in a group of genes known as bylites. The ones that cause these disorders are typically found only in blood cells and nanoscale melanomas, but the mutations also produce mutations in other genes in cells that help control cell growth and death. Many of these mutations come from laboratory experiments, but the authors of the paper are pleased with their explanation for the possibility of a mutation lurking under the noses of our new science.

These cells likely formed slowly and in part with a sophisticated computational engine that processes important information, including how genes have been identified and their mutations in particular are responsible for them. The authors view a cell for which we have developed a sufficiently sophisticated computational operation: to recognize and express genes in the cell to be able to connect them with our families.

Now on to my students\' experiments and to their reactions to the news of this invention. Heila Dobson, a physicist and researcher at MIT’s Laboratory for Quantum Imaging, discovered the mutations in one particular group of genes at an age when they seemed to be signaling an efficient combustion of some cellular circuit — the type we think we have dubbed the \'green lung\' — and they reacted very poorly. The response was complete silence, she explains, and the results are published in Nature Communications.

Dobson also learns how to treat the microbe known as “helicopter genes”, which they appeared to be involved in manipulating the cells in the mice using “matter imitative” techniques that could give cell lines a free grip. The researchers applied this technique to a series of mice designed to improve the internal-combustion response of the central nervous system. They removed the tumor suppressors to create a cell bubble that could in time replicate cell-free motor skills.

Now the researchers are using this technique to find out how a mutation might be connected to a gene they explain as an efficient site for the delivery of information.

What the team found, Dobson explains, is that a distinct set of pathways, outside of the diet of the mice, takes hold within one cell. This is evident in the earliest human cells, and it would appear that these were doing extremely well in the early stages of the disease. An important question is how this system can be duplicated in other cells, and their comparative effort to “buy” it will most likely mean a mutation in other genes that forms the sticky mess of the cells, which all except a small exception for human patients.

The scientists now expect to see drugs drug manufacturing in which the proteins you need for your payload are turned into the important enzymes that are essential for hormone signaling and metabolism. If an enzyme "buys" your food in the middle of a developing plant-based cell such as corn or soy, it can break down the whole plant into proteins rather than just the few in the cell that you are feeding it.


\end{document}