
\documentclass{article}
\usepackage[utf8]{inputenc}
\usepackage{authblk}
\usepackage{textalpha}
\usepackage{amsmath}
\usepackage{amssymb}
\usepackage{newunicodechar}
\newunicodechar{≤}{\ensuremath{\leq}}
\newunicodechar{≥}{\ensuremath{\geq}}
\usepackage{graphicx}
\graphicspath{{../images/generated_images/}}
\usepackage[font=small,labelfont=bf]{caption}

\title{Breast cancer remains very uncertain through the course of clinical}
\author{Nicholas Marshall\textsuperscript{1},  Chris Beltran DVM,  Jimmy Mendoza,  Emily Nguyen,  Jason Ware,  Donna Anderson,  Eric Kim}
\affil{\textsuperscript{1}Ewha Womans University}
\date{January 2012}

\begin{document}

\maketitle

\begin{center}
\begin{minipage}{0.75\linewidth}
\includegraphics[width=\textwidth]{samples_16_489.png}
\captionof{figure}{a woman and a man pose for a picture .}
\end{minipage}
\end{center}

Breast cancer remains very uncertain through the course of clinical trials, because of declining clear and qualitative survival rates in breast cancer. The prognosis in breast cancer remains poor, often due to chronic disease and percivitization – which involves repeated DNA sequencing and cell placement – once in remission – prior to mastectomy.

Researchers at a university in the southern province of Seoul have discovered that an inflammatory cancer cell (MCLC) exists that can confer good and unexpected prognosis on susceptible patients. This means that they have found a high-trending tumor cell that can block positive prognosis in the lymphatic system.

The newly confirmed cells are a result of improvements in DNA expression, specifically the activation of the multiple gate proteins known as MACRPs that control the cell’s responses to high levels of radioactivity.

An estimated 70,000 to 80,000 tumours are diagnosed worldwide each year and these will generate about \$9.6 billion in value for hospitals and health care facilities. The new research, led by scientists at the Institute of Molecular Medicine (IMM) in Seoul, brings the number of tumor cell cancer cells known to exist into the range of 21 candidates. The new cells, which are some 11 times more common than previously suspected, have different DNA structures (other elements and genetic profiles) than previously suspected.

This higher number of DNA samples than previously suspected was responsible for the emergence of cells that were also mutated, resulting in the discovery of cells more responsive to radiation, warfarin, etc. New DNA processing techniques may yield more consistent results for these cells as they are able to react with more complex signaling pathways, CNC diagrams, and other complexity and biological activity, not to mention manipulating little flow mutations and signals throughout the body that keep them alive.

The new cells may also assist clinicians and patients seeking the most effective preventive measures for breast cancer, such as treatment and medication awareness training, and clinical tests of chronic disease.


\end{document}