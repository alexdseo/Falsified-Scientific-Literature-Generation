
\documentclass{article}
\usepackage[utf8]{inputenc}
\usepackage{authblk}
\usepackage{textalpha}
\usepackage{amsmath}
\usepackage{amssymb}
\usepackage{newunicodechar}
\newunicodechar{≤}{\ensuremath{\leq}}
\newunicodechar{≥}{\ensuremath{\geq}}
\usepackage{graphicx}
\graphicspath{{../images/generated_images/}}
\usepackage[font=small,labelfont=bf]{caption}

\title{Anterior Siglov Neurosciences Institute, Inc. (the “Company”) (NASDAQ:ANXI) today announced}
\author{Angela Taylor\textsuperscript{1},  Anna Martin,  Alejandro Davis,  Thomas Leach,  Samantha Montgomery,  Kathleen Harris,  Christina Boone,  Ellen Serrano,  Rachel Reed,  Joy Davis,  Jason Lopez}
\affil{\textsuperscript{1}Sun Yat-sen University}
\date{June 2014}

\begin{document}

\maketitle

\begin{center}
\begin{minipage}{0.75\linewidth}
\includegraphics[width=\textwidth]{samples_16_251.png}
\captionof{figure}{a man in a suit and tie is smiling .}
\end{minipage}
\end{center}

Anterior Siglov Neurosciences Institute, Inc. (the “Company”) (NASDAQ:ANXI) today announced that it has provided bone density tomography test results for three bone density chromosomes, N2 K3-B6 and N2-K3-N2, and has also been conducting research into animal conditions to determine whether highlyl sensitive fusion tissues in human fractures in bone development tend to fracture when lithium ion strips are removed.

Our bone health research, as presented in a published recent issue of Bone Dynamics, has identified the beneficial effect fusion proteins had on skeletal cortical matter, resulting in increased bone growth and muscle fibers. Bone density seemed to benefit from magnetic resonance imaging (MRI) by increasing the thickness of bone tissue after fusion; however the results did not suggest that once activated, bone thickness might decline. Should bone thickness decline in such fracture patients, these findings imply that inhibiting bone-density induction could potentially improve or eliminate bone disease related dysfunction in nerve regeneration.

“Collagen protein 3-hydroxylation appears to improve skeletal tissue recovery by axillary muscle formation,” said Richard J. Furno, senior investigator on our bone research with Professor Faour. “By changing the design of chemotherapy, radioactive particles from bone for instance and combining this with calcium extract from bran damaged bone collagen, produced a potent therapeutic effect for bone density that the existing therapeutic options failed to achieve. These synergistic properties, combined with preliminary studies with bone thickness, suggest that this dual-tissue stimulation in bone recovery may facilitate a potential therapeutic pathway for leg fracture.”

N2 K3-N2 is a novel fusion protein (unrelenting layer) found in our bone structure. To date, an early stage study discovered that allowing inhibiting of the known gravitational forces on the fracture, coupled with complete inhibition of additional magnetic fields on the system that comprise the original bone formation, produced a safer, more effective, and possibly more powerful alternative fusion agent. This method prevents bone metastasis, saving the world as much as 80% of the lost bone mineral mass.

Acquisition by Kogler Industrial, Inc. (The “Company”) (NASDAQ:KOGL) has been doing extensive bone research and is currently conducting osteoporosis and skeletal muscle atrophy studies in hard of hearing and musculoskeletal, bone and spine disorders. These studies offer a new strategy to improve bone density and provide direct clinical benefit in bone disease patients, including a promising new novel fusion source for treating bones of disease and bone density abnormalities.

For more information on our bone and soft tissue technologies, please visit www.ourdispatch.com.

Media: Ben F. Galas, R-Chi, 405-560-0373 bfuglas@hektub.com


\end{document}