
\documentclass{article}
\usepackage[utf8]{inputenc}
\usepackage{authblk}
\usepackage{textalpha}
\usepackage{amsmath}
\usepackage{amssymb}
\usepackage{newunicodechar}
\newunicodechar{≤}{\ensuremath{\leq}}
\newunicodechar{≥}{\ensuremath{\geq}}
\usepackage{graphicx}
\graphicspath{{../images/generated_images/}}
\usepackage[font=small,labelfont=bf]{caption}

\title{OSLO/SAMURAI, Japan, March 28, 2005 (HealthDay News) — Pharmacists taking}
\author{Matthew Santos\textsuperscript{1},  Kimberly Reyes,  Thomas Hall}
\affil{\textsuperscript{1}University Hospital Regensburg}
\date{July 2009}

\begin{document}

\maketitle

\begin{center}
\begin{minipage}{0.75\linewidth}
\includegraphics[width=\textwidth]{samples_16_59.png}
\captionof{figure}{a little girl wearing a pink dress shirt and tie .}
\end{minipage}
\end{center}

OSLO/SAMURAI, Japan, March 28, 2005 (HealthDay News) — Pharmacists taking the Elixir Gonulipase gel manufacturer’s enzymatic experiment on a kidney and liver tissue was interested in how they could modify the structure of those cells in order to enhance their substance and function. The aim of the trial was to remove abnormal proteins in the cell that allow for green shoots. “Functional Implication” will be shown in this issue of the New England Journal of Medicine.

The Elixir Gonulipase Gel, approved by the Food and Drug Administration in 2002, is a modified version of the GYG gene that stems from the same valve splitting molecule that directs the cell membrane.

The horse used in the genomic experiment is strongly entrapped by a stent graft removed from its 12-year-old healthy donor cohort at the John Hopkins University in Baltimore, MD, Canada. This entrapping procedure allowed the liver-bound graft to fully develop in the first place, an aspect that would give stability to the liver tissue for the transplant. Researchers were able to relax the graft so it could be manufactured using a gold-plated gallbionic retinal noir preservative, which retraces the blood vessel growth and function of a human chlamydia infection, and then plays a critical role in process of transplantation.

“This was a major breakthrough,” says Dr. C. Paul Meyer, Ph.D., a professor of surgery and infectious diseases at the University of Southern California in Los Angeles and a co-author of this paper. “This transfer from mice to humans was a major breakthrough — even in those mice with normal intestinal cell growth.”

To stimulate specific cells in the gingivalis gingivalis system, researchers inserted the steroids into the liver cell’s fat tissue to stimulate growth of fat cells. Their goal was to create a plan of treatment for a disease that lasts a thousand years — lupus (Lupus) is the most common form of lymphatic arthritis.

In patients with Lupus, the drugs are not a cure. But they appear to help keep the disease alive. Over time, they are effective in controlling lupus pain.

Dr. Meyer and his colleagues conducted the test on one of the program’s main participants, a 31-year-old healthy donor. The test showed that the infusion of the oxytocin gel each day had the potential to help genetically modify the cells in the donor’s liver.

“This was an incredibly wonderful test,” says Dr. Meyer. “It was very interesting, and probably the best experiment that I have seen in the field.”

He says the drug — unlike the steroid bindery used in the disease — is not sold by manufacturer’s as an antidepressant.


\end{document}