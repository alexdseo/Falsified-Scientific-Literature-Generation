
\documentclass{article}
\usepackage[utf8]{inputenc}
\usepackage{authblk}
\usepackage{textalpha}
\usepackage{amsmath}
\usepackage{amssymb}
\usepackage{newunicodechar}
\newunicodechar{≤}{\ensuremath{\leq}}
\newunicodechar{≥}{\ensuremath{\geq}}
\usepackage{graphicx}
\graphicspath{{../images/generated_images/}}
\usepackage[font=small,labelfont=bf]{caption}

\title{Current study finds seven consistent phytoestrogenic patient expressors have a}
\author{Jessica Holt DDS\textsuperscript{1},  Emily Martin,  David Miller,  Stuart Patterson,  Carol Smith,  Amanda Turner,  Rachel Pierce,  Robert Page,  Eric Moore,  Bryan Whitney MD,  Casey Rice}
\affil{\textsuperscript{1}Blood Transfusion Centre of Slovenia}
\date{March 2012}

\begin{document}

\maketitle

\begin{center}
\begin{minipage}{0.75\linewidth}
\includegraphics[width=\textwidth]{samples_16_161.png}
\captionof{figure}{a woman is holding a teddy bear in her arms .}
\end{minipage}
\end{center}

Current study finds seven consistent phytoestrogenic patient expressors have a similar level of quality of life in patients with nephrotic syndrome.

A team of researchers from the UCLA School of Medicine\'s Comprehensive Cancer Center at the UC San Francisco School of Medicine, in collaboration with the Taiwan National Institutes of Health, and Chang Jin-sang, a professor of medicine, uncovered the differences between the phytoestrogenic patients with nephrotic syndrome and the standardised patients who responded to the previous methods.

The new findings follow a study last year from another UCLA affiliate that demonstrated many of the phytoestrogenic patients had better overall quality of life.

"By measuring the phytoestrogenic patients\' responses to diseases that focus on the protein \'pegocytopostat\' or ITC, you can evaluate the chances that the phytoestrogenic patients have less severe, chronic illnesses," Dr Qian-shong Zhu, director of the center\'s department of ophthalmology, and Dr Xue-Zhao, a study co-author, said in a release from the American Academy of Ophthalmology.

Vomiting after multiple visits with multiple phytoestrogenic patients in March 2006 made the results of the current study "remarkable", the authors write.

Phytoestrogenic patients that were on an ITC response rate of 80 to 90 per cent showed increased interest in the treatment in April 2007, the authors write.

Phytoestrogenic patients with a phytoestrogenic patients treatment declined by 1,750 milligrams per day, compared to a treatment rate of 30,000 milligrams per day for those with an ITC treatment for 35 days. The overall ratio of improvements between treatment and memory, respiration, respiration and respiration for patients treated with the PHYPT form was 622 to 943, and in any case they showed improvements according to symptoms.

The first study published on March 12 tracked 48 treatment of six adult patients. All observed reductions in episodes of diarrhoea, vomiting, dizziness, fatigue, joint pain, lower limb pain, cough, testiness, tinnitus and abnormalities in fibromyalgia, glaucoma, and in blood pressure, metabolism, and glucose tolerance.

"Despite being six weeks into their treatment, these patients were still receiving relatively mild to moderate disease-related viral loads, which appears to be of no consequence for chronic disease-related symptoms," the authors write.

The other treatments they assessed were viral polymerase and rubicon enzymes, antibiotics that did not require additional components to get rid of the enlarged cells, and medications that had the effect of reducing viral loads by 1,440 per cent.

They found the proportion of patients who had a ≥difference in the patient\'s responses to the first treatment with amprox-1, had declined by less than a half a percent.

However, they noted, the ratio of improvements to decreases by 2,400 per cent was half the rate of placebo versus the most common average in the study of 104 treatment patients with an MS or Alzheimer\'s disease with a known Positron T antigen (PTO-AR) gene gene.

"This illustrates that no treatment is 100 per cent effective when non-PTO-AR outcomes are chosen in the same group," the authors write.


\end{document}