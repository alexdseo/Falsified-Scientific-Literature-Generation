
\documentclass{article}
\usepackage[utf8]{inputenc}
\usepackage{authblk}
\usepackage{textalpha}
\usepackage{amsmath}
\usepackage{amssymb}
\usepackage{newunicodechar}
\newunicodechar{≤}{\ensuremath{\leq}}
\newunicodechar{≥}{\ensuremath{\geq}}
\usepackage{graphicx}
\graphicspath{{../images/generated_images/}}
\usepackage[font=small,labelfont=bf]{caption}

\title{Considering that the mythical scorpion is a "research film", one}
\author{Joseph Dalton\textsuperscript{1},  Matthew Riley,  Kelly Schmitt,  Rebecca Richmond,  Evan Campbell,  Stephen Chavez,  Molly Rodriguez}
\affil{\textsuperscript{1}University of Pittsburgh}
\date{April 2012}

\begin{document}

\maketitle

\begin{center}
\begin{minipage}{0.75\linewidth}
\includegraphics[width=\textwidth]{samples_16_95.png}
\captionof{figure}{a woman in a white shirt and black tie}
\end{minipage}
\end{center}

Considering that the mythical scorpion is a "research film", one would have expected that its pure behavior could be considered as evidence of magnesium sulfates

Magnesium sulfates, a mineral found only in rats and pigs and only in feces or rats. The researchers did research on this mineral in the lab using adult cows; it is illegal to carry out work on animals which contain magnesium sulfates. However, this product, which is unknown to humans, may be inactivate and give unsavory effects in certain animals.

A paper by the team, led by Dr. Chih-Zen Chang and Dr. Aij-Lie Kwan, at the Experimental Research Centre of Marvus Agricultural Institute in Medical Sciences University of Jammu-1 in India, says:

Magnesium sulfates have reportedly been found in animals for a time in humans, but then an enzyme in animals was given to them and discovered to be twice as potent in adult animals. Some of this magnesium sulphate is already known to have potential in soy. Knowing the magnesium form in animals used as an anti-inflammatory agent can help in building stronger immune systems in susceptible animals.

The discovery, if approved, may not lead to animal testing but could provide the scientific community with information to indicate if magnesium sulfates might also do the same in beneficial anti-inflammatory activities, the researchers say.

Lithospermate is a mineral which is much sweeter than magnesium sulfates, which is known to reduce inflammation and contribute to cardiovascular health. The researchers suspect that magnesium sulfates are supposed to only excrete calcium ions and enhance the absorption of calcium and promote neuron regeneration. Researchers believe that our magnesium sulfates are the only naturally occurring sources of sodium carbonate in mouse diets.

The publication of this study is published in the journal Davos.

The site is owned by Blue Prairie Emerging Technologies, a laboratory of 26 Indian laboratories.


\end{document}