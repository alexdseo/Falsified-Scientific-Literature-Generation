
\documentclass{article}
\usepackage[utf8]{inputenc}
\usepackage{authblk}
\usepackage{textalpha}
\usepackage{amsmath}
\usepackage{amssymb}
\usepackage{newunicodechar}
\newunicodechar{≤}{\ensuremath{\leq}}
\newunicodechar{≥}{\ensuremath{\geq}}
\usepackage{graphicx}
\graphicspath{{../images/generated_images/}}
\usepackage[font=small,labelfont=bf]{caption}

\title{Regulated Expression of the Beta2-Toxin Gene ( cpb2) in Clostridium}
\author{Timothy Martinez\textsuperscript{1},  Linda Brooks,  Sherry Webb,  Nathan Jennings,  Joshua Butler,  William Bates,  Sandra Wise,  Samantha Acosta,  Kenneth Smith,  Maurice Robinson,  Jeffrey Boyle,  Danielle Hunter,  Jennifer Campbell,  Jessica Mcclure,  Michael Lopez}
\affil{\textsuperscript{1}University of Tasmania}
\date{April 2012}

\begin{document}

\maketitle

\begin{center}
\begin{minipage}{0.75\linewidth}
\includegraphics[width=\textwidth]{samples_16_301.png}
\captionof{figure}{a man and woman posing for a picture .}
\end{minipage}
\end{center}

Regulated Expression of the Beta2-Toxin Gene ( cpb2) in Clostridium perfringens Type A Isolates from Horses with Gastrointestinal Diseases

RESULTS: E P to v2 P to v0

CONSTRUCTED MANY SYNDROME REMOVALS PROBABLY SAVVY

Biological and Scientific Evidence is presenting at the 18th Annual Fluid Medical Comprehensive Growth Study Annual Meeting, underway in New York on March 28, 2005, at the Boort/Kingston Institute.

Emerging data, highly funded and open-access, reflects CIM data from approximately 5,000 patients with liver disease in close contact with horses as therapy for treatment in live animals, primarily the canines of patients with chronic Type A infections, the horse side effects described in the Lancet in 2001 and the Animal Welfare Research Protection Guide, published in 2008.

Last year, 2.85 million people worldwide have affected by these human infections, most due to intestinal parasites such as tetranitium, known as the plague in humans. Further, tens of millions more people are seeking treatment for hepatitis and other liver diseases caused by alcohol abuse, smoking, obesity, HIV/AIDS, metabolic diseases such as hepatitis C and amyloidosis, and severe premature aging.

1) Most Type A Isolates treat peripheral nerves and stomach walls, with value in rural settings—nearly 1.5 million persons, nearly 50 percent of total, have suffered from torture and witnessing brutality in animal rehabilitation or detention. The exposure and deployment of torture is common, and perpetrators of torture receive some of the highest ranks of Human Rights Watch (HRW) fellowships and MIPC board appointments.

2) Sixty-five percent of deaths due to abuse or gangrene in the animal care systems are stemming from ongoing tortures involving horses. Horses killed, in bed and in various exercise facilities, by inhumane and degrading conditions are preferred for psychological exposure and providing therapeutic assistance.

3) In more than 30 percent of the animals in most U.S. U.S. livestock facilities, including trials in or on sale, captive testing environments, most cogs and horses are slaughtered. Equine training for animals undergoing and managed circumnavigation and confinement poses extraordinary risk of livestock violence.

FACTOR: Horse dieings are well documented and extremely uncommon. Sixty-four percent of estrhodopsemia, for example, occurs due to witnessing death in a herd, but mortality statistics are much higher. Likely cause of death is not discovered, which causes an inherent risk for developing certain metabolic defects and the metabolic disorders associated with human liver rejection, seizures, disease and brain damage. In most cases of severe endocrine dysfunction and hormonal dysfunction, the signs are too subtle to diagnose, leading to an exaggerated risk of heart disease and hypertension. Breed has also experienced other common eating disorder complications, such as stress hormones, and other metabolic disorders have a poor capacity to trigger cardiac and nervous system disorders.

4) Entomologists believe that veterinary practices increasingly cater to the behaviors of the horses themselves and that this new, "white-washing" approach is a cost effective way to protect these animals and their animals from cruel diseases.

Endocrine Disorder, overexertion, and extreme suction or "feedback" has been studied in animals resulting from consumption of nonstricture-induced or disordered nutrients. According to BGEB, an unproven family history study used to recruit more than 4,000 horses in pet condos in a Baltimore area, 11% or 1 in 3 died from wild horse nutrition. The study of overexertion, overexertion, and treatment of disease compromised the elderly, poor diet, and diet-associated psychological, physical, emotional, and behavioral problems. Very little is known about the risk of birth defects, metabolic problems, or seizures, and in starker contrast, even only 36% of horses with diabetic circulation die of certain diseases, most due to of disease that is not known to be previously associated with behavior modification. Furthermore, a common fate is death or injury. These studies record animals and their animals for only a small fraction of total gestational births, often occurring in or on sale.

5) Horse diets are known to be high in probiotics and low in animal-derived ractopamine, leading to serious complications, including severe anaemia in these animals.

As part of BGEB\'s captive testing, which held the study in single units for 4-7 weeks, our organizatio

\end{document}