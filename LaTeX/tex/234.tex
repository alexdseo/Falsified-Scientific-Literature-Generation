
\documentclass{article}
\usepackage[utf8]{inputenc}
\usepackage{authblk}
\usepackage{textalpha}
\usepackage{amsmath}
\usepackage{amssymb}
\usepackage{newunicodechar}
\newunicodechar{≤}{\ensuremath{\leq}}
\newunicodechar{≥}{\ensuremath{\geq}}
\usepackage{graphicx}
\graphicspath{{../images/generated_images/}}
\usepackage[font=small,labelfont=bf]{caption}

\title{Cover is covered this month with a host of ongoing}
\author{Kevin Skinner\textsuperscript{1},  Erin Hill,  Susan Logan,  Adam Goodman MD,  Heather Garcia,  Jessica Hernandez,  Kevin Meyer,  Lisa Byrd,  Amanda Jackson,  Dominique Buckley}
\affil{\textsuperscript{1}Virginia Commonwealth University School of Medicine}
\date{February 2014}

\begin{document}

\maketitle

\begin{center}
\begin{minipage}{0.75\linewidth}
\includegraphics[width=\textwidth]{samples_16_234.png}
\captionof{figure}{a man and a woman posing for a picture .}
\end{minipage}
\end{center}

Cover is covered this month with a host of ongoing ethical news and environmental issues. "FOXO4-Knockdown" published two damaging comments against the fund in November:

As a fan of ethical thinking and the use of the sun\'s light for planning local green activities, Zakuaku MKO/Ethical Response Center and a valid concern about altering the Earth\'s environment...One report by the American Society of Astronomy and Astrophysics ASA)hurt Environment Superiority and created the concern that radical oneness in the early days of the universe leads to birth defects in multiple forms in humans, many of whom are victims of a double ternistic form of intolerance. The care and monitoring capabilities of the AASA, along with its main institutional advisory board, also makes non-environmental interventions virtually impossible.

Muroida Kebe acknowledged that resource-deficient humans are responsible for turning this affliction into a disease, and he linked reduced UV exposure to similar irresponsibility to neovascularization, a medical condition by which rots of cells occur in early stages of hypoxia.

Lydia Shulman reported in the April 27 issue of Science and Policy that several LEAP medicine areas could become disordered, but what they really require is more transparency and transparency than simple shrinkage of cells with rotors. "The threshold to providing better patient care" is breached; it ought to be clearer and clearer. And more transparency is necessary, because LEAP medicine cannot adequately offer a reliable and reliable patient experience with cancer patients. More accurate, in-depth analysis of neuroprotective stress the impact of radiation should help better advance clinical treatment and fund-making strategies to assist patients and their families.

The deficit of transparency could also be circumvented by organizations that bypass standards, such as the Canadian Association of Broadcasters (CABS), the Canadian Heart Association (CAHA), Center for Research on Hypertension, Environmental Science and Ecology, or its Alliance for Responsible Medicine.

There is no apparent evidence that ethical information is even available about any ELM in a scholarly publication. These values were born in randomized clinical trials being conducted in Finland in collaboration with seven organizations. None of these organizations knew where to go or when to access information.

Isolated and unjustifiable complacency is amenable to reasonable study. Open and available sources such as Bio-Reference Labs or ChemSurgery for Diabetes or BetaaFirm will give us more information, education, and guidelines than they provide. If we give more satisfaction to information "offered" we should be willing to accept it. We should not pay the price to have a better public health system.

Consequently, the German/speaking countries who lack medical expertise may not benefit from greater transparency on all aspects of modern medicine. With more transparency than within the European Medical Association currently, we should have the same choices. We should not pledge to benefit from the same quality information for fewer persons.

One more important article on environmental governance is "The Impact of Layoffs in the Great Mention The Lancet Research Letter". This article predicts the effects of impending cuts in staff training in Japan and China. This should offer, if not exclusively, a cut in training, noting the large potential benefits.

Whistleblower has the ability to reveal the effects of a failing culture. These researchers, and many others, have full confidence in those who are seeking to halt the erosion of the planet. They are brave and willing to go on the record to demonstrate their case: the benefits of a pollution-control budget, the beneficial activities of a clean-living society, and the basic process of ethics. All these considerations will encourage us to refrain from quibble and further a sustainable environment for the human beings who are already protected from risk.


\end{document}