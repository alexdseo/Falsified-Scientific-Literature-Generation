
\documentclass{article}
\usepackage[utf8]{inputenc}
\usepackage{authblk}
\usepackage{textalpha}
\usepackage{amsmath}
\usepackage{amssymb}
\usepackage{newunicodechar}
\newunicodechar{≤}{\ensuremath{\leq}}
\newunicodechar{≥}{\ensuremath{\geq}}
\usepackage{graphicx}
\graphicspath{{../images/generated_images/}}
\usepackage[font=small,labelfont=bf]{caption}

\title{Researchers at the Federal Center for Vaccine Research at the}
\author{Linda Davis\textsuperscript{1},  Anthony Cain,  Christian Castillo,  Robert Myers,  Kara Reeves,  Sarah Campbell,  Jacob Wilson,  Mary Davis,  Daniel Jackson,  Samantha Potter DDS,  Ricardo Powell}
\affil{\textsuperscript{1}Mianyang Normal University}
\date{July 2009}

\begin{document}

\maketitle

\begin{center}
\begin{minipage}{0.75\linewidth}
\includegraphics[width=\textwidth]{samples_16_19.png}
\captionof{figure}{a man and a woman are posing for a picture .}
\end{minipage}
\end{center}

Researchers at the Federal Center for Vaccine Research at the University of California, San Francisco, discovered how methyltransferase (MT) excitation in the action of DNA carries positive ions to the molecular level in cancer cell communities.

Working with scientists at Mark Harris’ Stanford University Medical Center team, the team decoded hundreds of papers to create a program that could potentially aid diagnosis of any form of chemotherapy.

The work highlights the disalability of DNA methyltransferase inhibitors (LIPs) that would allow studies of curative effects of leukemias (e.g., PA, Erythropoietin); or other types of chemotherapy drugs. As the treatments become more expensive in terms of cost and different ways to treat, the likelihood of finding an acceptable treatment is diminished, and resources have to be diverted to other tumor types in order to treat specific types of cancer.

The research is an especially important advance in the fight against several carcinomas, according to the CDC, whose website explains that with the recent data from a study showing that gene methyltransferase inhibitors enhance the effectiveness of gene therapy in treating certain cancers, the need for a simpler development path to the FDA once consideration was made for generic versions has increased.

The study was published online in Journal of the American Medical Association on March 28, 2014.

Published in Cell Reports, the journal noted that human research has shown that “MTH is traditionally derived from nuclei from grasses that sit below the surface” and that the amount of methyltransferase in DNA has an effect not unlike that seen in water. Researchers in the Chemotherapeutic Substances at the University of California, San Francisco, discovered how methyltransferase induces a unique response in the action of DNA methyltransferase in cancer cells that has no ability to change the target protein upon activation of DNA methyltransferase, but could be achieved by using long-life biocontrol or having a yield booster.

These findings showed a number of promising targets for making new treatments for tumors. For example, the current study also found that methyltransferase indicated an improvement in remission after induction of the drug into patients with non-Hodgkin’s lymphoma. We have known for more than a decade that methyltransferase would interact with other types of chemotherapy drugs, and it appears that for human trials, an expanded drug bar model could be developed.

\#\#\#


\end{document}