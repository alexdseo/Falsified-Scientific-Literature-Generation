
\documentclass{article}
\usepackage[utf8]{inputenc}
\usepackage{authblk}
\usepackage{textalpha}
\usepackage{amsmath}
\usepackage{amssymb}
\usepackage{newunicodechar}
\newunicodechar{≤}{\ensuremath{\leq}}
\newunicodechar{≥}{\ensuremath{\geq}}
\usepackage{graphicx}
\graphicspath{{../images/generated_images/}}
\usepackage[font=small,labelfont=bf]{caption}

\title{Our world and natural environment are threatened by the interplay}
\author{John Mack\textsuperscript{1},  Kelsey Jackson MD,  Scott Calhoun,  Thomas Rojas}
\affil{\textsuperscript{1}Saint Joseph University, Lebanon}
\date{June 2004}

\begin{document}

\maketitle

\begin{center}
\begin{minipage}{0.75\linewidth}
\includegraphics[width=\textwidth]{samples_16_76.png}
\captionof{figure}{a woman in a dress shirt and a tie}
\end{minipage}
\end{center}

Our world and natural environment are threatened by the interplay of natural, industrial and global threats. Using a combination of sensors, advanced computer vision, and random recognition, now is the time to identify and kill culprits.

There are a lot of hard-to-detect chemicals that cause industrial accidents but not yet known to cause poisoning. There are many cases in which people may be unconscious or dying as a result of an industrial accident. These deaths include some known victims of near-misses through a contaminated site that contaminated the air, the transport system or the earth. Currently more than 1,200 people die in industrial accidents. With advances in sensors, computer vision, and random recognition of industrial risks the number of people exposed to poisoning will grow.

How do you do this?

Today we have a team of scientists and researchers using sensors to identify and treat industrial pollution. The team has developed an effective way of screening pollution for chemical poisons and is doing it under international treaties.

We start with a sensor that assesses the pollutants in a narrow area. Then we understand the chemical threshold and type of poisoning. Then we select the contaminant and it is observed in a silver cloud. If the contaminant is a specific type of chemical or other toxic ballast (biological)- such as mercury and glyphosate, we can detect them. This sensor triggers a chemical signature that enables us to examine the mist that spreads around the contamination region.

We have partnered with the U.S. EPA and the European Organization for Toxicology (ETA) in Estonia to work on this work. This was a first of its kind and they have now developed and published a potential prediction as to the next chemical level for poisoning. We are now ready to use the Optima peptide biological identification tool that we used to see the chemicals in mine workers and the infrastructure that infected them. If they are an unauthorized manufacturing facility then we are now ready to detect poisoning. This detects poisons that are expecially toxic with narrow spots and creates a biomarker that will be used on contractors if there is an attack.

Where to look: That is where the Environmental Protection Agency says different substances are discarded in the air before they hit the ground. The safest way to interpret those is to check that the contaminant in the air meets or exceeds the FDA level. However, there are some heavy metals going in the air and we expect that will also build up in the droplets of aerosolized fluid. So to compound this, we now introduce a new level of particles and small particles that can theoretically be used in the coatings when manufacturing pollutants.

Once the contaminants have been identified, they can be checked against the standardized standard of poisoning. By changing the composition of the product, we have the opportunity to sample the contents of the plant in order to understand the substance. By this point in time we need the ETA to validate the correlation between chemical exposure to hazardous air and the pre-existing toxicity of industrial substances. This will improve the management of industrial poisoning.

How do you compare these tools?

The initial benefit of using a combination of sensors and computer vision is to identify potentially hazardous substances. There are more examples of sites using chemical pollution detection that exist only in factories but there are many cities where textile and seafood factories connect to the industrial or chemical exhaust sources of industrial pollution. Our current goal is to find pesticides, rather than chemical or chemical activities.

The potential of our work is that the Algonquin Environmental Network (AENE) can help us identify hazardous materials that could enter rivers, creeks, or human settlements and kill someone else. AENE makes possible the communities along which rivers and creeks come together to keep the environment healthy and accessible.

While the sensors can help to identify hazardous substances, we use natural tools such as de-stabilizing pyrolysis (re: grading), chemical avoidance and filtration, phasing, tracex-system testing, and other proven system of detection. The technologies used to test materials are sometimes difficult to detect and solve with a technical eye, but the critical materials that are made are the raw materials used. Most of these industry depend on our ability to identify positive chemicals that could contaminate human food or water in the presence of human exposure.


\end{document}