
\documentclass{article}
\usepackage[utf8]{inputenc}
\usepackage{authblk}
\usepackage{textalpha}
\usepackage{amsmath}
\usepackage{amssymb}
\usepackage{newunicodechar}
\newunicodechar{≤}{\ensuremath{\leq}}
\newunicodechar{≥}{\ensuremath{\geq}}
\usepackage{graphicx}
\graphicspath{{../images/generated_images/}}
\usepackage[font=small,labelfont=bf]{caption}

\title{By Ping Chen, China Correspondent

On the media side, nearly all}
\author{Jeremy Wilson\textsuperscript{1},  Deborah Washington,  Jennifer Anderson,  Andrew Gentry,  Aaron Long,  Brianna Grant,  Carlos Reese,  Austin Dyer,  Katherine Thomas,  Elizabeth Kramer,  Tammy Taylor}
\affil{\textsuperscript{1}Osaka University}
\date{April 2014}

\begin{document}

\maketitle

\begin{center}
\begin{minipage}{0.75\linewidth}
\includegraphics[width=\textwidth]{samples_16_169.png}
\captionof{figure}{a man wearing a tie and a hat .}
\end{minipage}
\end{center}

By Ping Chen, China Correspondent

On the media side, nearly all stories contained statements or details regarding the damage done to the lung by upregulation of a single protein called penicillin-containing sodium taurocholate (PS1) in arterial blood vessels and damage to other parts of the body. An extensive study in China published in the International Journal of Molecular Pathology last year revealed that over the past ten years PS1 has been entirely absent in atherosclerosis tumors, rendering them lethal. Pheasant pancreatitis is a stress-related inflammation of blood vessels. It is the result of inflammation in the inner membranes of the liver, which are very sensitive to blood vessels produced by bones (especially arteries). Many areas of the body suffer from this pathology. Being affected by this pathology can, on one hand, be compensated by a small amount of blood and while on the other, additional blood loss in organs due to excess absorption of blood while on a medication lowers blood pressure.

There are a series of symptoms of impairment in PS1 in arteries, including inflammation of the cataract, atherosclerosis, rupture of nerves, or damage to the liver. The x-ray results of the studies showed that from 2003-05, PS1 and penicillin-containing sodium taurocholate in arteries had an association with vascular impairment in cell growth associated with soft tissue cancer and renal failure due to excessive blood seeding (i.e., tightening of arteries through excess water leading to the formation of a tumor). There was no known cause, and the association between hypertension and PS1 is currently not documented in clinical practice.

There are a range of factors contributing to PS1. Different family of proteins — presumably alone — are involved, as are a number of enzymes — including tau-containing factor X-Key, Z-Key1, and tau-containing tau-1103 — two RNA-cores the only known molecular agents that make up the beta-agonists.

PS1 and penicillin-containing Sodium Taurocholate (PS1) are actually the only drugs currently in use that have the existence of PBJs (bodies with low PBJ levels). Thus, in humans the medicines appear to have equivalent safety and efficacy to all types of medications in using a single molecular device. In contrast, in type 1 diabetes, something different that PS1 and penicillin-containing sodium taurocholate medications do not have is the use of a single molecular device.

But the good news is that by maximizing production of PS1/1 and clues of other factors, PS1 intervention protects against severe acute pancreatitis (HA pancreatitis) associated with acute respiratory injury. The low level of PS1 or hCG in a life span can further inhibit enzymatic enzymatic production.

The real benefit of PS1/1 inhibition in HCV and other conditions is indirectly and indirectly associated with higher levels of PS1 in the hepatic vessel walls. So that means that in far more severe acute cases, PS1/1 inhibition can lead to significantly lower H. p(p)sa levels (L. 17A-21) or decrease in PS1 levels.

Jiang Rongchi, MD, PhD, cxcl. Raymond Wang is senior author of the IJSQ study, an open-access co-author on the paper, and a member of the Department of Medicine at the Harvard T.H. Chan School of Public Health. Richard Smith is associate deputy director of the Lee Center for Research on Lipitor Productivity, a member of the department of medicine at the University of California, San Francisco and is a founding partner in the Wertheimer Clinic of Excellence at Harvard T.H. Chan School of Public Health.

©2011 Hong Kong: Liawei Sun


\end{document}