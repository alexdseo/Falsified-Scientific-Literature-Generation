
\documentclass{article}
\usepackage[utf8]{inputenc}
\usepackage{authblk}
\usepackage{textalpha}
\usepackage{amsmath}
\usepackage{amssymb}
\usepackage{newunicodechar}
\newunicodechar{≤}{\ensuremath{\leq}}
\newunicodechar{≥}{\ensuremath{\geq}}
\usepackage{graphicx}
\graphicspath{{../images/generated_images/}}
\usepackage[font=small,labelfont=bf]{caption}

\title{Taking Analog Astrid Raunuar as one of the four "endless"}
\author{Douglas Carpenter\textsuperscript{1},  Donald Carlson,  Kevin Cruz,  Andrew Johnson,  David Peterson,  Michael Velez,  Brian Harper,  Rhonda Lopez,  Charles Massey,  Anthony Hernandez,  Carrie Wilson,  Mckenzie Pena}
\affil{\textsuperscript{1}Johns Hopkins University}
\date{January 2014}

\begin{document}

\maketitle

\begin{center}
\begin{minipage}{0.75\linewidth}
\includegraphics[width=\textwidth]{samples_16_180.png}
\captionof{figure}{a woman in a white shirt and black tie}
\end{minipage}
\end{center}

Taking Analog Astrid Raunuar as one of the four "endless" treatment options such as surgery, prosthetics, medication, and therapy for brain and spinal deformities can raise serious health risks for people with pre-existing conditions, according to an article published in AVA News by Shyam, Wwale Achi, Gabiragandabi, and Vasifulkan Dulya.

Source: Focuses Intense Biomedical Realization Program Photo / Shutterstock.com

What to do When Depression Oils are Improving Brain Health

Dr. Turgay Dulya, a Neuroendocrinologist at Johns Hopkins University School of Medicine who wrote the article, explained that unabsorbed stress and inflammation (or inflammation caused by a low-level stress hormone that promotes stress in the brain) is the culprits behind depression.

According to Dulya, the stress that leads to depression may worsen the brain’s risk for developing brain problems. For example, after you take medication, the boost of serotonin may impair cognitive function.

Drugs can cause depression, but in many cases it doesn’t clear up for a few months. For this reason, treating depression involves using the usual therapies – mindfulness, medication to control inflammation, strategies to think critically, and other medical advice.

Here’s an example from Shyam of some of the treatments he’s been using:

“We have tried all sorts of things like yoga and I’ve even tried antidepressants. But when you take out all the social connections that you can really build in the brain, you begin to lose that link to be able to help your emotional and mental health. It’s just going to increase the number of rashes and inflammation. You can then look at another avenue of being optimistic about your state of mind.”

Admittedly, this is not by definition the only reason for depression. Attention-caring patients might actually have an affected mind, and so on.

In the article, Shyam explains how he never gives up on those things to help his patients avoid feeling overwhelmed. He explains that four barriers prevent him from caring for depression:

Fear and so forth. The difficulty of assuming that depression doesn’t leave a lasting impact in the lives of people with basic needs, like getting up and moving, sleeping, and doing things. The idea that when the potential loss of these connections are eliminated, the depression is gone; there is no future left.

“Physical reactions. The physical response to my depression is biochemical, and I can see the metabolites which are involved to make this come back. This helps me to say that I’m hurting and that I need you to get over it.”

When depression doesn’t disappear; you only expect to feel sad and depressed. Unfortunately, these subtle ways in which you interact with your own mental health can affect your cognitive and mental health.

In more clinical studies, antidepressants have been linked to high levels of depression, though what is more, taking a tiny amount of medication can increase your chances of alleviating symptoms.

That should be good news for people who might be experiencing depression: There is plenty of medication out there that can help people reduce anxiety and depression symptoms.

You can find the following tips for controlling inflammation from Shyam of these four treatments:

Examine your mood at home. It may be time to get your nervous system checked. What medication is you taking? Getting rid of toxic chemicals that can interfere with your blood pressure and blood flow, like GPs, your gynecologist, and your physician can do that.

Get the Financial Treatment of Loss

A small sum of money or even a few pounds of living expenses will dramatically reduce your depression.

Have you experienced one of the four "endless" treatment options such as surgery, prosthetics, medication, and therapy for brain and spinal deformities? Please share your experiences with us on the obesity \& diabetes section.


\end{document}