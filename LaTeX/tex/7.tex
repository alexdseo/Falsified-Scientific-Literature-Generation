
\documentclass{article}
\usepackage[utf8]{inputenc}
\usepackage{authblk}
\usepackage{textalpha}
\usepackage{amsmath}
\usepackage{amssymb}
\usepackage{newunicodechar}
\newunicodechar{≤}{\ensuremath{\leq}}
\newunicodechar{≥}{\ensuremath{\geq}}
\usepackage{graphicx}
\graphicspath{{../images/generated_images/}}
\usepackage[font=small,labelfont=bf]{caption}

\title{Heart

miR-1915 and miR-1225-5p Regulate the Expression of CD133, PAX2 and}
\author{Andrew Wright\textsuperscript{1},  Joseph Schultz,  Ariel Nicholson,  Robert Odonnell,  Amanda Carter,  Kurt Oconnor,  Vincent Wright,  Karen Flores,  Jill Palmer}
\affil{\textsuperscript{1}Concordia University Ann Arbor}
\date{March 2011}

\begin{document}

\maketitle

\begin{center}
\begin{minipage}{0.75\linewidth}
\includegraphics[width=\textwidth]{samples_16_7.png}
\captionof{figure}{a woman in a dress shirt and tie holding a cell phone .}
\end{minipage}
\end{center}

Heart

miR-1915 and miR-1225-5p Regulate the Expression of CD133, PAX2 and TLR2 in Adult Renal Progenitor Cells Delayed in 2011

By Gaiki Marandino \& Gabriela Chantel

During the 10th day of March 2013, an international scientific body, the Registry, in collaboration with the U.S. Department of Justice (DOJ), announced that the authorities from the FDA, Eulogy-Justice (Franklin), and the European Union had put in place a recommendation for a review on the language of CD133 in adult non-adult biological processes related to adult oestrogen.

The risk of developing a mutation in the language of CD133 and resistant to those controls are low, and the ability to develop the expression of CD133 in adult neurons is limited, but the risk of a tumor progression in adult neurons is high.

The investigators presented the results of the European Commission’s 14-day post open access access study, conducted by the European Commission and European Association of Infant Neonatal Medicine, at the Annual Meeting of the European Academy of Sciences in Frankfurt, Germany, March 17-20, 2013.

The authors report that a decision by the FDA, Eulogy-Justice (Franklin), and the European Union member states to delay notification of the preliminary investigation results is troubling. More alert consumers should contact the Agency for Healthcare Research and Quality (AHRQ), an agency appointed by the FDA, when the preliminary investigation results are released. The full text of the study can be found at http://www.nationalastroethicatory.com/site/complete.pdf

About the European Commission

The European Commission is one of the leading regulators of mental health and public policy initiatives in the EU. With more than 6,000 members, the Commission has empowered over 22,000 policymakers and professionals to act in the public interest in the development of standards and guidelines, and implement legislative and strategic actions. The Commission is part of the European Union and shared responsibility for setting standards and requirements for innovation, performance and access to healthcare services. The Commission is committed to a design of a coherent EU framework to promote global solutions to ensure universal healthcare and increase the level of access for all people.

http://www.ft.com/news/2018/01/29/dpa-u-ag-develops-a-new-by-guarding-academia-deal-worldwide.html

About the WHO

The World Health Organization (WHO) is the universal healthcare organization (U.N. agency for the treatment of people with serious and life-threatening illnesses) in the 21 countries of the world which also comprise the European Union. WHO enforces sustainable universal healthcare and health systems through its various global missions and international cooperation in health, family planning, nutrition, and safety and health in general. Worldwide, it undertakes its leadership and advocacy activities to protect human rights, promote universal healthcare, and protect children from acts of violence and their rights to life. With more than 32 million health workers, we have a unique role to play in meeting the needs of the global community for comprehensive healthcare; medical conditions, treatment and care, education, employment, poverty alleviation, health-related services, migration and others. The WHO also offers vital services such as the division of labor between developed and developing countries.

Reference

\# \# \#

1 'Plano-3', paragraph 6 Fabio Sallustio and Gabriela Chantel

2 'Qaix-6, MediPerro'

3 'J-2', paragraph 8 Gaiki Marandino and Gabriela Chantel

4 'Revista-6', paragraph 12 Georehrle Weinberg

5 'Yesum'

6 'True'


\end{document}