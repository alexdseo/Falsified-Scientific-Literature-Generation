
\documentclass{article}
\usepackage[utf8]{inputenc}
\usepackage{authblk}
\usepackage{textalpha}
\usepackage{amsmath}
\usepackage{amssymb}
\usepackage{newunicodechar}
\newunicodechar{≤}{\ensuremath{\leq}}
\newunicodechar{≥}{\ensuremath{\geq}}
\usepackage{graphicx}
\graphicspath{{../images/generated_images/}}
\usepackage[font=small,labelfont=bf]{caption}

\title{Scientists at Stanford University’s IARC Comprehensive Cancer Center in San}
\author{Marissa Rodgers\textsuperscript{1},  Sarah Munoz,  Cody Conley,  Amber Morgan,  Megan Warner,  Fred Frazier,  Julia Hudson}
\affil{\textsuperscript{1}Ecole Normale Superieure, Paris}
\date{March 2014}

\begin{document}

\maketitle

\begin{center}
\begin{minipage}{0.75\linewidth}
\includegraphics[width=\textwidth]{samples_16_328.png}
\captionof{figure}{a man in a suit and tie holding a microphone .}
\end{minipage}
\end{center}

Scientists at Stanford University’s IARC Comprehensive Cancer Center in San Francisco have linked ICS-Epiphenase inhibitor IPC proteasome inhibition with reduced immune response in C C lymphocytes.

Led by study co-author Prof. Simone Fabre, UC Santa Barbara, the groundbreaking findings showed that ICS-Epiphenase inhibitor IPC proteasome inhibition ultimately improved the quality of C C lymphocytes. The mechanism of cell division is an important factor behind the reduction of immune response.

C C lymphocytes are vital for immune system formation. Recent studies have demonstrated that ICS-Epiphenase inhibitors are a safe and effective anti-cancer-fighting tool that could potentially reduce the severe side effects of ICS-Epiphenase inhibitors in adult patients with C C lymphocytes, as well as C lymphoma patients with rare types of lymphoma.

“The findings from this study were robust,” said Professor Fabre. “Our work shows that C C lymphocytes reduce anti-cancer effects of ICS-Epiphenase inhibitors. Conversely, the molecules that inhibit the inflammatory responses of C C lymphocytes have no effect on L-cells.”

Preclinical studies from 2011 and 2012 found that the high dose (25 mg/kg/s) of ICS-Epiphenase inhibitors was associated with less severe adverse events associated with C C lymphocytes compared to similar cardiovascular events. C lymphocytes also respond well to ICS-Epiphenase inhibitors and demonstrated differentiation against induced T-cell activation, antiviral responses and apoptosis.

In ICS-Epiphenase inhibitors, the mechanism of cell division is usually determined by C C lymphocytes that are cultured from a litter of C C lymphocytes. The order in which myCS-Epiphenase inhibitors penetrate C C lymphocytes suggests the potential protection of targeted therapies against C C lymphocytes.

In this study, C C lymphocytes were cultured from a litter of ICS-Epiphenase inhibitors and were labeled with a particle that had formed into a progenitor compound. These results offer only limited context. The cells were a batch of unrelated DNA that had known to be extracted from their cell cousins. Within 60 days, every one of the cells produced enough messages that the cytotoxic toxic substances on their cells appeared. The target molecules on their cytotoxic substances spontaneously cooled to excitatory zones, thereby lowering cholesterol and increasing heart and kidney function. MyCS-Epiphenase inhibitor repulsed the methanogens by blocking C C lymphocytes by catecholone calcium ions with IV-GA-120 and acetanes and then ultimately eliminated them. This signaling mechanism is part of the mechanism responsible for the decrease in immune response and excretion of C C lymphocytes, as well as the higher rates of heart and kidney inflammation associated with ICS-Epiphenase inhibitor therapy.

The authors write:

“Our findings were exciting to us because they uncover the hard, yet difficult mechanisms of cell division. It gives us new insights into the mechanisms of myCS-Epiphenase inhibition and drives therapeutic intervention and therapeutic paradigms that cannot be tackled by standard treatment regimens.”

To view the authors’ abstract for their study, view the video for the abstract here or read the full paper here.


\end{document}