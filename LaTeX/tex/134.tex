
\documentclass{article}
\usepackage[utf8]{inputenc}
\usepackage{authblk}
\usepackage{textalpha}
\usepackage{amsmath}
\usepackage{amssymb}
\usepackage{newunicodechar}
\newunicodechar{≤}{\ensuremath{\leq}}
\newunicodechar{≥}{\ensuremath{\geq}}
\usepackage{graphicx}
\graphicspath{{../images/generated_images/}}
\usepackage[font=small,labelfont=bf]{caption}

\title{Breast cancer metastasis may require targeted radiolabeling to gut cells}
\author{Ray Wilson\textsuperscript{1},  Samuel Martinez,  Brenda Horne,  Aaron Rivera,  Christopher Potts}
\affil{\textsuperscript{1}Saga University}
\date{August 2011}

\begin{document}

\maketitle

\begin{center}
\begin{minipage}{0.75\linewidth}
\includegraphics[width=\textwidth]{samples_16_134.png}
\captionof{figure}{a man in a suit and tie holding a cell phone .}
\end{minipage}
\end{center}

Breast cancer metastasis may require targeted radiolabeling to gut cells that share a mechanism for the mutation development of cystic fibrosis mutations. Mitochondrial cancer, which is the most common form of cancer, spreads when chemotherapy therapies fail to bind directly to the cancer cell.

This research presents the first science-based paradigm-change to date that has enabled cancer researchers to reveal a metabolic receptor mutation that could transform the DNA of multitudes of cancer metastasis into responsible pathways to disease-free tissue.

Cancer metastasis affects some 15 percent of the population worldwide, and the diagnosis of metastatic disease has the potential to be a cause for remarkable rifts in biological and chemical systems.

This research demonstrates the case for targeting cancer cells specifically targeted to ultra-small intercellular neuromyositis epithelial cells (IMNPS). NSF-funded Uev1A-Ubc13 is a novel, and active and transgenic, radioactive-derived scaffold that eliminates the potential for over 80 percent of the classical agent mutations in breast cancer.

The NSF-funded Uev1A-Ubc13 system is a thyroid-specific version of the protein-environmentally cleared ibits-2 gene transfer and transgenic modified aztafinase (ILC) protein for the delivery of transgenic lines of an ultra-small hydrolysis of alpha- and beta-gal growth factor (PGGF).

For comparative analysis, Weiwei Zhang, MD, of UCSF, and his colleagues created the Uev1A-Ubc13 system using pcyclonomy.ai, a genetic repair mechanism that ties messenger proteins called kinase-2 with the major character receptors of fibroids in the human body, and cancer cells. In contrast, the Uev1A-Ubc13 system is unique in that it uses a specific specific pathway of viral sepsis and transgenic mice that function as molecular interconnecters.

The NSF-funded Uev1A-Ubc13 system was developed by a Uev1A-Ubc13 team and the Obagi and Chen Foundation in partnership with the Fujian Research Institute. Weiwei Zhang

“This is a truly breakthrough advancement that impacts breast cancer metastasis into cystic fibrosis biology, but it does so at very, very small scales with far less potential discovery and opportunity to save lives,” said Weiwei Zhang, M.D., assistant professor of Medicine and Biochemistry and Whitehouse professor of Molecular Pharmacology and Development at the UCSF School of Medicine.

Furthermore, Weiwei Zhang and other Uev1A-Ubc13 researchers also discovered a novel mechanism of action in nodules called gene-transferforming molybdenum-1, which is their ideal target for cancer-friendly intermediate regulatory signaling structures. In addition, they found a role in the adaptation of NPCM proteins to bladder fibrosis.

The mouse study, led by Weiwei Zhang, MD, was published in the journal Molecular Cancer and was supported by grant HS419225 and the National Cancer Institute of the United States Cancer Institute, Cancer Genome Connection, Chatham-Kent Cancer Institute, Gannett Cancer Research Institute, Singapore Grant Thornton, Acadis Research Foundation, School of Medicine and Biochemistry, University of Southampton, in the United Kingdom.

“Nuclear factor-free connectivity of prostate cells has led to novel new approaches in genomics to identify tumors and to develop next-generation biomarkers,” said Weiwei Zhang, MD, a professor of Medicine at UCSF and a member of the US Molecular Cancer and Development Institute. “This finding, which published in Molecular Cancer, serves as a guide for this molecular transformation of prostate cancer.”

Nuclear-level-associated mRNA modifications that support the cellular population would also be useful for cancer-friendly intermediate regulatory signaling structures,” said Weiwei Zhang, MD, a major member of the UCSF-funded Uev1A-Ubc13 committee.

Allowing cancer cells to share the expression of the proteome-binding protein pathway involved in metastasis will certainly reduce cancer production and spreading disease, and decrease the value of tests to alter or reprogram the cell environment to achieve targeted molecular control.


\end{document}