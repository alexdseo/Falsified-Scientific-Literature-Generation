
\documentclass{article}
\usepackage[utf8]{inputenc}
\usepackage{authblk}
\usepackage{textalpha}
\usepackage{amsmath}
\usepackage{amssymb}
\usepackage{newunicodechar}
\newunicodechar{≤}{\ensuremath{\leq}}
\newunicodechar{≥}{\ensuremath{\geq}}
\usepackage{graphicx}
\graphicspath{{../images/generated_images/}}
\usepackage[font=small,labelfont=bf]{caption}

\title{Press Release

With an estimated 30 million Americans reporting stiffness, stiffness,}
\author{Randall Adams\textsuperscript{1},  Courtney Thomas,  Justin Martinez,  Robin Lucas,  Jose Taylor,  Erin Watson,  Regina Robinson,  Mallory Roman,  Dr. Kristine Parks,  Jamie Allen II}
\affil{\textsuperscript{1}Harvard University}
\date{June 2008}

\begin{document}

\maketitle

\begin{center}
\begin{minipage}{0.75\linewidth}
\includegraphics[width=\textwidth]{samples_16_476.png}
\captionof{figure}{a man and a woman posing for a picture .}
\end{minipage}
\end{center}

Press Release

With an estimated 30 million Americans reporting stiffness, stiffness, back stiffness, trauma, or whatever for mild reactions, one side effects of electroacupuncture treatments that lead to endoscopic paralysis can be devastating. Vascular pain occurs when the smallest area of myelin that surrounds the spinal cord is cut into, and resulting in paralysis. Currently, there are no effective clinical treatments that adequately activate these effects, and in the last decade, there have been fewer experimental treatments and surgery for these patient-specific symptoms. Such treatments include serenity-based alternatives (extended limbs, or aged limbs) and some noninvasive approaches.

“I am very excited to learn the role of electroacupuncture in enabling the development of nerve pain,” says Jim Kispert, M.D., clinical professor of neurology at Columbia University Medical Center in New York, who received the prestigious Freeman Spogli Institute Prize in Medicine for the study of electroacupuncture.

The NYU professor’s trial involves three medulloblastoma-raphed mice, and their corresponding symptoms include stiffness, back, and neck pain. Chronic spine pain is often associated with paralysis. Serenity-based treatments commonly approved in the 1980s may contain anachronistic electrical energy that controls the movement of myelin. Magnetic resonance imaging (MRI) imaging has shown that electrical signals may cause a localized sensation in the paralyzed area, where there is no spontaneous muscles to clear the axons into place.

Kispert’s brain investigation turned to electroacupuncture as an alternative to vasoconstriction therapy, currently used in many acupuncture clinics. One study showed in 150 patients, that patients with cerebral-neutered pain experienced an electrical response to electroacupuncture that caused an electrical nerve to run loose. The action was followed for three years, before the affected patients returned to the residency clinic.

Kispert’s treatment involves the insertion of electrodes into the bladder and using magnetic field currents to bind nerves. These free-floating electrodes are added to the brain and stimulate the nerves. In many other studies, electrodes are attached to nerve-reversing laser miasules. Initial results from the EEG experiment are promising, showing that research demonstrating electroacupuncture is effective in the brain has been advancing significantly since late 2013.

In an editorial, Associate Dean Sun Qi, M.D., S.B.C., director of the department of otolaryngology and anesthesiology, says that the evidence is “weak to establish that electroacupuncture brings out an abnormal region of the nervous system” or “is an indicator of the severity of our neuromuscular movement problem.” Moreover, it may be a possible cause of “a lot of pain-related problems in the body,” he adds.

Limitations of the trial include lack of clinical guidance from academic and clinical organizations that are the primary stakeholders for the research. With this type of trial, important scientific and medical information is missed. “All the physicians and scientists making this trial are in a position to provide this study to their patients,” says Shapiro.

For more information on electroacupuncture, please contact Dr. Mona Wang, Medical Director, NYU Langone Medical Center.


\end{document}