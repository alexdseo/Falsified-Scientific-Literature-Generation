
\documentclass{article}
\usepackage[utf8]{inputenc}
\usepackage{authblk}
\usepackage{textalpha}
\usepackage{amsmath}
\usepackage{amssymb}
\usepackage{newunicodechar}
\newunicodechar{≤}{\ensuremath{\leq}}
\newunicodechar{≥}{\ensuremath{\geq}}
\usepackage{graphicx}
\graphicspath{{../images/generated_images/}}
\usepackage[font=small,labelfont=bf]{caption}

\title{“The chemical reactions involved in the formation of tumor tissue}
\author{Ashley Ho\textsuperscript{1},  Terry Graham,  Stephanie Meadows,  Billy Eaton,  Lisa Taylor,  Timothy Kelly,  David Nicholson,  Doris James,  Barbara Pineda,  Linda Hughes}
\affil{\textsuperscript{1}INFN - Istituto Nazionale di Fisica Nucleare}
\date{January 2012}

\begin{document}

\maketitle

\begin{center}
\begin{minipage}{0.75\linewidth}
\includegraphics[width=\textwidth]{samples_16_191.png}
\captionof{figure}{a woman in a red shirt and a red tie}
\end{minipage}
\end{center}

“The chemical reactions involved in the formation of tumor tissue or human cells may have remarkable repercussions on cancer survival and metabolism," said Scott Hodge, M.D., Dean/Founder of InterMune Medical Systems, an Israeli company involved in a novel system for transferring two major human-derived versions of our own tumor drug molecules (derby and oxybioactive receptor-8901) into target cells. "These metals, described in the principle papers of Micron\'s Electrolysis Unit, serve as an enzyme and cytotoxic agents. PET and MDL (transforming polymerase-2) have shown the ability to induce tumor growth, mutate and overcome degrading toxicity within tumor cell membranes."

UPCHUJELTA,000.000 1.The chemical reactions involved in the formation of tumor tissue or human cells may have remarkable repercussions on cancer survival and metabolism. "The chemical reactions involved in the formation of tumor tissue or human cells may have remarkable repercussions on cancer survival and metabolism," said Scott Hodge, M.D., Dean/Founder of InterMune Medical Systems, an Israeli company involved in a novel system for transferring two major human-derived versions of our own tumor drug molecules (derby and oxybioactive receptor-8901) into target cells. "These metals, described in the principle papers of Micron\'s Electrolysis Unit, serve as an enzyme and cytotoxic agents. PET and MDL (transforming polymerase-2) have shown the ability to induce tumor growth, mutate and overcome degrading toxicity within tumor cell membranes."MORTALIZATION has been shown to produce damaging alterations in living cells, making molecular modifications or torts invasive at an advanced stage. The microRNAs, shaped inside cells with spinal cord, kidney, and liver fibrotic-type membranes, act as a critical sensor for DNA transcription factors, signaling pathways, and cell proteins. MicroRNAs are found in tissues from multiple patients, blood, bone marrow, and within tumors.MicroRNAs protect us from many diseases that are otherwise under investigation, either by governments, doctors, or forensic laboratories, and present a much greater threat to our daily lives. With a few, cost-effective drugs, tumor agents are becoming available to the laboratory, which now more quickly establishes the human-derived drug target than any other factor in our daily lives. These microRNAs are found in tissues from multiple patients, blood, bone marrow, and within tumors. MicroRNAs protect us from many diseases that are otherwise under investigation, either by governments, doctors, or forensic laboratories, and present a much greater threat to our daily lives. With a few, cost-effective drugs, tumor agents are becoming available to the laboratory, which now more quickly establishes the human-derived drug target than any other factor in our daily lives. However, there are several limits on when microRNAs are present, which is a debate that continues in the scientific field. However, there are several limits on when microRNAs are present, which is a debate that continues in the scientific field. Little is known about the character of microRNAs: whether they are an energetic production molecule, or an infectious state. Thus, whether microRNAs are a factor, or an infectious state. In response to criticism from the ethical and scientific community over past questions regarding microRNAs, however, the late Synapse Laboratories in Nairobi, Kenya, published their own genome sequencing data in a publication entitled, \'Is there no substance to the question of microRNAs?\' (Publication - 2009). Synapse Lab confirms the microRNAs\' broad role in everything from tumor activation and microRNA manipulation to the disease-causing processes involved in decision-making in cancer treatment. However, there are several limitations on when microRNAs are present, which is a debate that continues in the scientific field. However, there are several limitations on when microRNAs are present, which is a debate that continues in the scientific field. In response to criticism from the ethical and scientific community over past questions regarding microRNAs, however, the late Synapse Laboratories in Nairobi, Kenya, published their own genome sequencing data in a publication entitled, \'Is there no substance to the question of microRNAs?\' (Publication - 2009). PERUNDI (for Metastatic and Chronic Inclusive Brain Diseases) The findings reveal that the NMDA’s New Biogen Idec Phenotypic Metastatic Ingredient (NBI) inhibitor, applied to breast and colon cancer cells, has the ability to increase th

\end{document}