
\documentclass{article}
\usepackage[utf8]{inputenc}
\usepackage{authblk}
\usepackage{textalpha}
\usepackage{amsmath}
\usepackage{amssymb}
\usepackage{newunicodechar}
\newunicodechar{≤}{\ensuremath{\leq}}
\newunicodechar{≥}{\ensuremath{\geq}}
\usepackage{graphicx}
\graphicspath{{../images/generated_images/}}
\usepackage[font=small,labelfont=bf]{caption}

\title{Scientists from the U.S. and Europe have identified an essential}
\author{Adam Alexander\textsuperscript{1},  Raymond Watts,  John Willis,  Katherine Riddle,  Aaron Clay,  Jesus Delgado,  Beth Ibarra,  Nicole Mcdonald,  Daniel Rhodes,  Karen Brown,  Gina Robles,  Douglas Jones,  Kimberly Kim}
\affil{\textsuperscript{1}Columbia University}
\date{July 2014}

\begin{document}

\maketitle

\begin{center}
\begin{minipage}{0.75\linewidth}
\includegraphics[width=\textwidth]{samples_16_45.png}
\captionof{figure}{a man and a woman posing for a picture .}
\end{minipage}
\end{center}

Scientists from the U.S. and Europe have identified an essential role for inositol and EGFR in a multidisciplinary pair of oral proteins – the reticulated version of gut bacteria.

The growing range of genetic markers in human genes and tissues has allowed scientists to observe interactions of genes in intricate moments in life, making them very useful for identifying specific proteins in our bodies.

The finding – that injecting human genetics and animal animal DNA into natural sequence of gut bacteria helps to identify metabolic metabolites at a crucial stage in life – is important for developing biomarkers for the identification of key metabolites in our physiology that help to fight off disease.

To learn more about the finding, scientists at USC Columbia and the University of Alabama, Birmingham, published their research published in ESO Proceedings of the National Academy of Sciences on February 25 in the journal Science Advances.

The study has implications for other areas of human genetics, such as gene expression and communication between genes and as the human genome has evolved to include hundreds of genes. In addition, scientists have further identified the pathways or regions that enhance long-term evolution, such as the ability to manipulate a molecular clock. The pivotal findings reveal that analyses of DNA trace data from organisms examined in the first-ever dry prospective lab assay have a significant role to play in genetic decision-making.

“This breakthrough in human genetics in the synthesis of erythropoietin and high value-added RNAi sequences is an important discovery that will impact our understanding of diseases of the very earliest stages of life, including today’s postnatal, postmenopausal and post-menopausal women,” said Daniel K. Kander, MD, PhD, a member of the USC USC CSIRO Academic Center for Genetics and Molecular Medicine.

Source: USC CSIRO

Physicists Identify Essential Role for Inositol Pyrophosphates in Type-I Interferon Response


\end{document}