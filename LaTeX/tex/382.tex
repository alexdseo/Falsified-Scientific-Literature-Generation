
\documentclass{article}
\usepackage[utf8]{inputenc}
\usepackage{authblk}
\usepackage{textalpha}
\usepackage{amsmath}
\usepackage{amssymb}
\usepackage{newunicodechar}
\newunicodechar{≤}{\ensuremath{\leq}}
\newunicodechar{≥}{\ensuremath{\geq}}
\usepackage{graphicx}
\graphicspath{{../images/generated_images/}}
\usepackage[font=small,labelfont=bf]{caption}

\title{Physicians have today created a new tool to validate the}
\author{Karen Turner\textsuperscript{1},  Kelly Smith,  Christopher Martinez,  Richard Vance,  Luis Johnson,  Steven Sullivan,  Linda Taylor,  Michael Jackson}
\affil{\textsuperscript{1}Tsinghua University}
\date{August 2008}

\begin{document}

\maketitle

\begin{center}
\begin{minipage}{0.75\linewidth}
\includegraphics[width=\textwidth]{samples_16_382.png}
\captionof{figure}{a man and a woman are standing together .}
\end{minipage}
\end{center}

Physicians have today created a new tool to validate the role of early and late gene expression in the development of immune cells. These findings, performed in a new cellular immunotherapy device, revealed a novel mechanism by which the earliest CCL2 expression occurs in secondary immune cells, while the primary mRNA expression is minimized in primary human monocytes.

"The amount of actual CCL2 expression that an embryo expresses during the human DNA lifecycle (transportation of CCL2 materials to the cellular endoscope) was almost 5 times higher compared to protein expression, but much of the initial expression did not occur by genetic manipulation. The new cellular immunotherapy device to validate the study brings this hypothesis to light, allowing potential new therapies." said Dr. Mark Hirschson, senior author of the research study and a fellow at the General Hospital of the US at University of Arizona.

To study test the mechanisms by which mRNA expression reaches the molecular level of the CCL2 expression level, Dr. Andrew Chan, then the principal investigator on the work, established an HIV-specific vaccine gene called BRCA1932. The patient\'s active BRCA1932 gene was reduced in the CCL2 expression quantity and allowed for the inclusion of one or more of his body\'s immune cells. In addition, BRCA1932 gene expression was reduced in HIV-specific protein expression (DNAGLB), where the levels of the CCL2 expression quantity were also lowered. This expansion of the antibody was not mediated by other proteins. BRCA1932 gene expression was targeted to a protein called BRNAs (NDEs), allowing an immune response against BRCA1932.

The idea is that, in conjunction with a spike in CCL2 expression, the production of CCL2 in secondary immune cells, plus CCL2 in core adult monocytes, explains Dr. Chan\'s team of investigators.

The mechanism in question was discovered in the sequencing of six TOCATE1 (which was recently found to be a CCL2 expression agent) tests. The accumulation of CCL2 in the sample in the plasma of human monocytes leads to a solidified CCL2 expression program and a long-term synthesis of CCL2, shown in Tables 2 and 3.

"This was a good opportunity to make some new clinical applications. When patients are cured of the toxicity of their therapeutic disease, a clinical application could be made to test for a potential therapeutic pathway," says Robert Kuperberg, a University of Arizona

Added Dr. Kuperberg: "Both a CCL2 expression based on mRNA expression and an mRNA expression produced from the central cell nucleus (NKKC) has recently been published in the Nature Genetics Journal. Their findings underscore the potential for genome editing to discover new therapeutic mechanisms."

Follow us on Twitter @cnnbusiness.


\end{document}