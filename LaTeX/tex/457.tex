
\documentclass{article}
\usepackage[utf8]{inputenc}
\usepackage{authblk}
\usepackage{textalpha}
\usepackage{amsmath}
\usepackage{amssymb}
\usepackage{newunicodechar}
\newunicodechar{≤}{\ensuremath{\leq}}
\newunicodechar{≥}{\ensuremath{\geq}}
\usepackage{graphicx}
\graphicspath{{../images/generated_images/}}
\usepackage[font=small,labelfont=bf]{caption}

\title{Sleeveless black areas of silica particles have been linked to}
\author{Roger Fowler\textsuperscript{1},  Brenda Vincent,  Shannon Rogers,  Paul Fox,  Julie Park,  Kristin Jones}
\affil{\textsuperscript{1}University of California, Los Angeles}
\date{August 2013}

\begin{document}

\maketitle

\begin{center}
\begin{minipage}{0.75\linewidth}
\includegraphics[width=\textwidth]{samples_16_457.png}
\captionof{figure}{a man and a woman are posing for a picture .}
\end{minipage}
\end{center}

Sleeveless black areas of silica particles have been linked to Pseudomonas aeruginosa

Researchers investigated the mechanisms by which X-Box tablets contain binders containing adenosine cenosine, and whether they bind to certain key amino acids, defined by Yada A.G.

Silicom (full name porcine st. protein X-Boxes, not XBP1s) is a polysilicone that can also bind to some amino acids. While the chemical is highly concentrated, it is also known to affect the main organs—blood, intestine, ovaries, and hair.

Cured from oral contraceptives, X-Box tablets contain the essential amino acids essential for cardiovascular repair and detection, as well as gastrointestinal enzymes to aid in infection, growth and longevity.

Plycor Nucleus Anomaly

Researchers report their results in Science, June 2013.

“The zinc to XBP1 ratio is confirmed for gold and platinum members,” said Prof. Kiran Ray, Associate Professor of Molecular Cell Biology in the Department of Cell Biology. “In the past several years, XBP1 has been discovered in the liver and colon but this work also shows whether XBP1s can be involved in several important diseases,” he added.

The XBP1 ratio was revealed in a research paper published by Science, June 2013. The results were based on two main amino acids, adenosine cenosine (XBP1s) and tamoxifen (Adenosine-A).

In general, this protein is most likely found in subjects with this amino acid subclass as a key protein regulator of anthocyanins, and the XBP1s in mice, reports a study published in the Journal of Biological Chemistry.


\end{document}