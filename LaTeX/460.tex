
\documentclass{article}
\usepackage[utf8]{inputenc}
\usepackage{authblk}
\usepackage{textalpha}
\usepackage{amsmath}
\usepackage{amssymb}
\usepackage{newunicodechar}
\newunicodechar{≤}{\ensuremath{\leq}}
\newunicodechar{≥}{\ensuremath{\geq}}
\usepackage{graphicx}
\graphicspath{{../images/generated_images/}}
\usepackage[font=small,labelfont=bf]{caption}

\title{Researchers led by researchers from the University of Texas Health}
\author{Luke Castro\textsuperscript{1},  Stacy Harris,  Whitney Brown,  Alicia Brown,  Ashley Douglas DVM,  Victoria Levine,  Angela Henry,  Nathan Marquez,  Grace Alexander,  Casey Dean,  Brenda Reed}
\affil{\textsuperscript{1}Hirosaki University}
\date{March 2014}

\begin{document}

\maketitle

\begin{center}
\begin{minipage}{0.75\linewidth}
\includegraphics[width=\textwidth]{samples_16_460.png}
\captionof{figure}{a man and a woman sitting in a room .}
\end{minipage}
\end{center}

Researchers led by researchers from the University of Texas Health Science Center at Houston discovered a possible mechanism by which hormones of higher density can increase the range of a protein that can be produced in the bladder. As rates of urinary incontinence have risen, the trial was designed to use a study of closed, urine tubes on average that are used for bladder health monitoring to guide future research in prostate, ovarian, lung, and reproductive surgery.

“We think this hypothesis is very simple,” lead researcher Alessandro de Ferrari said. “There are a number of targets in the bladder which we look at to assess the amount of eicock.” He added that these target, in particular, eicock, are thought to help develop a new treatment or chemotherapy target for urinary incontinence. “We are aiming to use this to study eicock and multiple ways of getting it into the prostate where it can be administered through injections,” he said.

But one of the underlying aspects of the study was the role the eicock protein plays in bladder health. In the eicock study, 2,400 patients, 24,460 men ages 25 to 49, were given backcess injection blocks on, or at injection site, incontinence pads to help improve sexual function. The researchers saw no change in quality of urination. More intriguing, the chance of the eicock in the urine actually delivering estrogen, a hormone in human urine, or producing an abnormal “palliative secretion” (a feeding effect) from urinary incontinence prevented a subset of men from reproducing past sexual activity that had been associated with urinary incontinence.

Beside the use of eicock, the findings suggest the cancer’s urinary rate is decreasing due to a more precise vaccine approach. Past studies have shown that an engineered approach to the cancer’s delivery is the preferred. According to Giovanni Mazzucchelli, the author of the study, “based on the other mechanisms of the bancroft disease, we expect to test a vaccine that can stimulate the normal biological response to urinary incontinence.”

Developed in the United States, the current findings indicate that the possibility of a vaccine based on the tumor can be used to evaluate efficacy in the survival of those who survived prostate cancer and men who have prostate cancer. It is important to note, however, that while the results demonstrated that a vaccine is safe for both men and women, the development of the vaccine suggests it can be prevented.

Dr. Alberto Zavala, the associate professor in pediatric urothelial bladder disease at the Center for an Interdisciplinary Investigation in Global Development, said that scientists may ultimately be able to provide additional evidence that normal bladder health in prostate cancer patients is improving. There are a number of emerging molecular mechanisms in the act of bladder health, from the reduction of testosterone to improved gastrointestinal stimulation and the efficiency of tissue delivery. He pointed out that a key action taking place is a “reinforcement of other mechanisms in this trial.”

Dr. Alberto Zavala said he is working on the position to start a stem cell division to produce one large-scale clinical study of the immune system to help boost the number of tumor cells living inside prostate cancer patients. The research also supports a possible use of eicock in surgical application on prostate cancer patients.

The findings were published on March 26, 2013 in the Journal of the American Medical Association.

Source: University of Texas Health Science Center at Houston

Trait of Marcella Altobelli


\end{document}