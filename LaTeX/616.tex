
\documentclass{article}
\usepackage[utf8]{inputenc}
\usepackage{authblk}
\usepackage{textalpha}
\usepackage{amsmath}
\usepackage{amssymb}
\usepackage{newunicodechar}
\newunicodechar{≤}{\ensuremath{\leq}}
\newunicodechar{≥}{\ensuremath{\geq}}
\usepackage{graphicx}
\graphicspath{{../images/generated_images/}}
\usepackage[font=small,labelfont=bf]{caption}

\title{By Sarah Donohue

Scientists have discovered a new way to inhibit}
\author{Michael Frank\textsuperscript{1},  Katherine Campos,  Dr. Michael Kelly Jr.,  Eric Moreno,  Gregory Caldwell,  Richard Byrd}
\affil{\textsuperscript{1}Universiti Sains Malaysia}
\date{April 2013}

\begin{document}

\maketitle

\begin{center}
\begin{minipage}{0.75\linewidth}
\includegraphics[width=\textwidth]{samples_16_402.png}
\captionof{figure}{a man and a woman posing for a picture .}
\end{minipage}
\end{center}

By Sarah Donohue

Scientists have discovered a new way to inhibit and trigger the production of oxaliplatin - a protein that supports human cancer and inflammation - through a9b1 integrin.

Osteoporosis involves inflammation of the joints, blood vessels and brain, causing inflammation in many parts of the body. Many of these complications can also be avoided by a9b1 inhibitor.

Nucleoside control, or N-drugs - in which multiple drugs are successfully combined to suppress the behavior of the immune system - is a major factor in overcoming or even recovering the disease. The N-drugs involved in blocking the amofacyls and monocytes (an immune protein responsible for triggering inflammation), include solanezumab, tramadol and lanthanumab.

Researchers discovered a new proteome for the control of the debilitating ALM (Complex and Integrated Molecule Dimitriynastinases on Folate) event at UC Davis: in mice whose body had a 2.0% oxygen-containing PH-9beta enzyme blocked the release of circulating PH-9beta molecules.

The study was led by Lamin Feagin, M.D., PhD, with the Steve Ostrovsky Center at UC Davis: Furuhi, Gai Rus, PhD, and Hock Chun, PhD. It is the first paper to demonstrate the synergistic effects of partnered clinical treatment on immunomodulation of the drug-specific ALM, which is the most common mutated protein in human disease.

The team found that the combination of human ALM (EGP) and the PH-9beta protein reduced dopaminergic activity in blood vessels after the treatment of non-blood mice with ENO (adult thyroid hormone inhibitor) treated with ENO 1 by 41% (split between group 50 and 75 depending on the subgroups) and then increased distribution of PH-9beta protein in breast tissue.

The duo also investigated the potential of substituting the combined mutant and EGP treatment with actual human chemotherapy (n=25) in early-stage cancer.

In animal models of human disease the combination of the two treatments in combination on the original EAE wasn't included, however, in some tumor models showing significant reductions in the toxicity of human treatments. The team showed that the combined treatment led to a marked reduction in toxicity and an accelerated growth rate in tumor species with detectable PH-9beta sites in the cancer patients.

Previous experiments with human breast tumors had shown that combination of healthy volunteers and patients with no cancer-associated proteins had a much better response to treatment than the anticoagulant as they had to initially slow their disease to a minimum of 10% production of EGP. Thus, the results provided a signal that patients were consuming more PH-9beta than previously thought and showed that such treatment reduced toxicity and ultimately impacted the tissue growth of the metastatic tumor.

The investigators attribute their findings to the fact that human drug-specific drug-specific agent delivers signals rather than chemicals. Their finding suggests that scientists can explore and further penetrate the therapeutic application of protein therapies in our real world environment.

The study was published online February 27, 2013 in The Lancet Medical journal.


\end{document}