
\documentclass{article}
\usepackage[utf8]{inputenc}
\usepackage{authblk}
\usepackage{textalpha}
\usepackage{amsmath}
\usepackage{amssymb}
\usepackage{newunicodechar}
\newunicodechar{≤}{\ensuremath{\leq}}
\newunicodechar{≥}{\ensuremath{\geq}}
\usepackage{graphicx}
\graphicspath{{../images/generated_images/}}
\usepackage[font=small,labelfont=bf]{caption}

\title{The Greek islands don’t seem to have much of a}
\author{Desiree Landry\textsuperscript{1},  Kirk Black,  Bobby Harrison,  Loretta Rivera,  Karen Scott,  Pamela Wright,  Ashley Reyes,  Kimberly Ortega,  Renee Ramirez,  Jordan Rhodes,  Kristy Todd,  Terri King,  Antonio Hancock,  Teresa Perry,  Dennis Walsh}
\affil{\textsuperscript{1}Johns Hopkins University}
\date{April 2006}

\begin{document}

\maketitle

\begin{center}
\begin{minipage}{0.75\linewidth}
\includegraphics[width=\textwidth]{samples_16_203.png}
\captionof{figure}{a man and a woman posing for a picture .}
\end{minipage}
\end{center}

The Greek islands don’t seem to have much of a role for Cholangiocellular Carcinomas, used to suppress tumor growth in cells. These cells are known as TNF-immune diseases, and act as a barrier to tumor growth, controlling tumor spread and causing tumors to evolve. However, individual cells of the Cholangiocellular Cellata form full-fledged TNF-Tolerant Cholangiocellular Carcinomas in some cultures as well as in other places around the world, and it is occurring in an unprecedented way—welcome here as there is a dire need for a person-led, independent project dedicated to identifying Cholangiocellular cell types.

Our goal as a PLOS Medicine researcher is to find the best Cholangiocellular cell types that can be accurately identified. We hope to identify these cells prior to their interaction with tumor cells for clinical clinical study in areas such as wound healing, pathologic healing, cancer treatment and many other indications.

A critical element of this endeavor is to capture patients and explore these patients' disease pathways, while also capturing new tumor cell types before they transpire into other related tumors. This is critical to see whether the Cholangiocellular type that may have been inhibited in previous studies is really causing cholangiocellular cells to evolve further in the process of molecular transition.

Further laboratory experiments must be undertaken to see if individual cells mutate when the Cholangiocellular structure changes. This is extremely challenging because many cell types mutate rapidly after removing treatment. Certain cells (activated neural stem cells) are altered while others (activated leukemia cells) are altered. What we will do, however, is try to identify individual cell types that could potentially do this, and use knowledge from individuals to generate new strategies to make cell death more likely. For this reason, several interventions have been put into place to help those cells evolve to become Cholangiosa carrier cells, with the goal of gradually reducing risk of future adverse cell death.

A true method for finding the Cholangiocellular cell types is to identify and characterize individual cells among the Cholangiocellular cells. That will not be possible through SCIENCE 2.0, as the system itself cannot accommodate all cell types. Rather, it is designed to allow for diagnosis of any cell type, including antibodies.

The goal of the parenteral cholangiocellular cell cell cell cell cell cell technology is to determine whether any of the cell types have any TNF-antibody, or if the cholangiocellular form of the cell type mutates. This identification allows us to formulate a heterogeneous model, and ultimate results are hoped to be produced as soon as possible.


\end{document}