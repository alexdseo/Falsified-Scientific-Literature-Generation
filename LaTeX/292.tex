
\documentclass{article}
\usepackage[utf8]{inputenc}
\usepackage{authblk}
\usepackage{textalpha}
\usepackage{amsmath}
\usepackage{amssymb}
\usepackage{newunicodechar}
\newunicodechar{≤}{\ensuremath{\leq}}
\newunicodechar{≥}{\ensuremath{\geq}}
\usepackage{graphicx}
\graphicspath{{../images/generated_images/}}
\usepackage[font=small,labelfont=bf]{caption}

\title{Researchers at Peripheral Genetics Corporation have discovered an unusual mutation}
\author{Darren Morris\textsuperscript{1},  Ashley Wilson,  Nicholas Shepherd,  Mrs. Nicole Barr}
\affil{\textsuperscript{1}Kurume University}
\date{January 2012}

\begin{document}

\maketitle

\begin{center}
\begin{minipage}{0.75\linewidth}
\includegraphics[width=\textwidth]{samples_16_78.png}
\captionof{figure}{a man in a suit and tie holding a baby .}
\end{minipage}
\end{center}

Researchers at Peripheral Genetics Corporation have discovered an unusual mutation in the antimalarial CYP-816-54 , which has been implicated in the uncontrolled exposure of mammalian blood molecules to single environmental therapies known as transit pathways. This study could mean an effective treatment for amyloidosis with precise doses or resulting in preventive heart valve replacement (MET (methanol and total oxygen), which is a long-term target of the drug therapy Dupixent. This is an untested therapy but it should be developed to stop amyloidosis and an alternative treatment for systemic risk poisoning by bacteria. This is discovered in Molecular Human Immunologist/Oncologist and Prof. Yang Fuzhou, Emeritus Professor of Chemotherapy and Immunobiology at the University of California Medical Center-Yun Ma Company, Dr. Huang Hua, Prof. Liu Peng, Prof. Wu Chang, Prof. Wang Jing, Prof. Cheng Chang, Prof. Veng Ling Xie, Prof. Yo-Wei Lo, Prof. Yang Shuqi Li and Prof. Zheng Zhao.

Journal of Molecular Life Science finds the CYP-816-54 mutation in the medicine Dupixent signal is responsible for the uncontrolled and possibly fatal deficiency of human blood permeine scaffolding and proteases resulting in an increase in amyloidosis risk in mice. Alterogenes.org (July 30, 2012, 6:51 AM, 7:11 AM) reports that INAP-640027, tianga mouse model acquired by the University of Hawaii-Hawaii cancer research team in April 2007. The mouse model returned with 1/64th of the mutated cancer antigen on chromosome 159, this implies the presence of an MRCMB mutation. A MacD/Toxoplasma peptide+DE3 / TGF combination was applied to the modified mice. The study suggests that low levels of TGF inhibit receptors for precursor proteins, testosterone, estrogen, and tyrosine kinase, needed to function normally and androgenics need to respond positively to acetylcholine, a functioning gene. They were able to monitor and confirm that the mutations in MRCMB activation on a placebo group. Their findings are published in the recent peer-reviewed journal Sino-Anchostomy.


\end{document}