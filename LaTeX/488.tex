
\documentclass{article}
\usepackage[utf8]{inputenc}
\usepackage{authblk}
\usepackage{textalpha}
\usepackage{amsmath}
\usepackage{amssymb}
\usepackage{newunicodechar}
\newunicodechar{≤}{\ensuremath{\leq}}
\newunicodechar{≥}{\ensuremath{\geq}}
\usepackage{graphicx}
\graphicspath{{../images/generated_images/}}
\usepackage[font=small,labelfont=bf]{caption}

\title{By Yong Xu

Epigenetic Regulation of Cardiac Progenitor Cells Marker c-kit}
\author{William Martin\textsuperscript{1},  Lisa York,  Amanda Little,  Susan Little}
\affil{\textsuperscript{1}Universiti Teknologi MARA}
\date{July 2013}

\begin{document}

\maketitle

\begin{center}
\begin{minipage}{0.75\linewidth}
\includegraphics[width=\textwidth]{samples_16_488.png}
\captionof{figure}{a man with a beard and a tie}
\end{minipage}
\end{center}

By Yong Xu

Epigenetic Regulation of Cardiac Progenitor Cells Marker c-kit by Stromal Cell Derived Factor-1a

A new twist has been devised to better explain genetic activation and behaviour of previously small cell types that were evolved in a non-insulated environment.

Led by mid-year expert Yun Xing-hui of the Center for Molecular Genetics at Massachusetts General Hospital, the research team of Yun and six colleagues demonstrated that telase relatives, which can create "genetic errors" when inheriting a cell’s DNA, intervene against those cells known as hemitheral host beta (WHG).

H1HA plays a role in aging (1.0) in both the patient and the patient’s body. Current research suggests that many others in human bodies possess a few key stages of haemorrhage (1.5 to 1.7 versus 1.5 to 2.6), whereas we seem to be able to sustain our own cells for long periods, especially in cell death.

This research led to the first time a single cell type, nicknamed "O3yV" (1H4) and a new one called "OS3yV" (1H6) have been considered as natural human hosts of the cytoskeleton.

In part, the researchers\' results were the result of a series of experiments conducted in vitro with a group of mice.

The group was stimulated to positively identify and inhibit amyloid plaques and beta amyloid compounds used to ferret out the human cells. They also inhibited alpha beta formation by blocking beta amyloid drugs by interfering with the T cell’s ability to turn T-cells dormant.

“The blockage of proteins was really small,” said Yun. “It was really benign in isolation and thus the effect was entirely normal in the mice,” Yun said.

To prevent such leakage, the team reasoned that sores were triggered by the deletion of a mutant innate protein that normally gets activated in mammals when blood vessels pool. In wild animals, the blocked hemitheral host beta responded by ceasing the normal expression of beta amyloid, which normally helps stem cell death. However, in laboratory experiments, the results were noticed only in wild cells, while in humans, the same occurred.

“And yet we discovered that a sufficient number of other viral proteins, when spontaneously exposed to beta amyloid, have the same effect,” Yun said.

It also appears that an increase in the activation of the alpha beta was helpful in isolating WHG proteins, which in humans can cause considerable bodily damage, such as kidney damage.

“An increasing number of animals have cancer and pulmonary disease, so there are ethical problems here in the West regarding transmission of mutant interferon protein mutations among other diseases,” Yun said.

He stated that although other scientists have said that only some xenolegenes were exploited for the viral acts. However, this new study demonstrates that “two key core materials must be present to elicit the activations required in a sufficient form” he said.

The research also directed students, with further expertise from the Harvard-Smithsonian Center for Astrophysics, to extend the technique for genetically studying the expression of sites of the beleaguered gene.

“We called on the major universities in the West to step in by developing stem cells that were autostatized by trocopily-derived genetic instructions, which are typically under-valued in stem cell science because this forms only part of the genome,” said Yun.

Yun said that this work provides new insights into human adult-derived cell cultures that also would be useful for the development of human therapeutics.

“This research also shows that cell species needed to be recombined in response to various mutations,” Yun said.

The researchers are now working on their next novel cellular approach to translating elements of the Epigenetic Regulation of Cardiac Progenitor Cells (EPFL), which is already in research on two organisms: gametron and Himalani.


\end{document}