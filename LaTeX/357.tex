
\documentclass{article}
\usepackage[utf8]{inputenc}
\usepackage{authblk}
\usepackage{textalpha}
\usepackage{amsmath}
\usepackage{amssymb}
\usepackage{newunicodechar}
\newunicodechar{≤}{\ensuremath{\leq}}
\newunicodechar{≥}{\ensuremath{\geq}}
\usepackage{graphicx}
\graphicspath{{../images/generated_images/}}
\usepackage[font=small,labelfont=bf]{caption}

\title{By Yuko Yoshida for Lung cancer

The study from Japan shows}
\author{Brandon Barker\textsuperscript{1},  Nathaniel Mendez,  Louis Thompson,  Jennifer Maldonado,  Michael Jones,  Kevin Porter,  Casey Murphy}
\affil{\textsuperscript{1}Tehran University of Medical Sciences}
\date{July 2001}

\begin{document}

\maketitle

\begin{center}
\begin{minipage}{0.75\linewidth}
\includegraphics[width=\textwidth]{samples_16_357.png}
\captionof{figure}{a man in a suit and tie holding a cell phone .}
\end{minipage}
\end{center}

By Yuko Yoshida for Lung cancer

The study from Japan shows that a common forms of medical marijuana use that are reserved only for patients with severe tumors, is associated with a decreased methylation of an important gene that controls the expression of a drug called “microRNA-34b” – a type of cancer gene that is genetically dependent on one type of microRNA called NADI, found in the urine of non-melanoma skin cancer.

The importance of microRNA-34b for cancer metabolism, the larger adult microRNA region that is responsible for the enzyme activity of lysosomal substances and the increasing rate of dependence of the substances, is not completely known but, perhaps surprisingly, it is important, particularly for a growth strategy, says the Department of Pharmaceutical and Biotechnology of NHK in the United Kingdom.

Almost 11 percent of all adult cancer in the UK uses drugs that become known as methylates, making them “more potent than other drugs”, says Peter Robinson, Royal Society President and Chief Investigator of the Skin Cancer Center.

Co-authored by Yoshida Tanaka and Kumiko Nishida.

UK study

The team was led by Professor Akira Tsujimoto, researcher at Cancer Research UK. The interest for doing this research arose on the ground in January 2008 and 2008 when the department published a study identifying a 25 percent decrease in the frequency of daily exposure to methylated DNA (MD) in the skin of men and women with melanoma, some 485 out of 1,000 melanoma patients. This was soon followed by a 25 percent decrease in MD which was subsequently confirmed by analyses done by the University of South Florida in January 2009 and as yet unconfirmed, but now known as National Early Detection of Cancer 2018, which was published in the journal Cancer Research.

Co-authors include: Nakai Matsuzaki, Professor at the Ronald McDonald School of Medicine and a Research Associate, at the Health Sciences Laboratory at the Interdisciplinary Center, Hiromasa Yamamoto, Shimori Ueno, postdoctoral researcher in Prof. Yasuo Takayama, senior scientist at the Institute of Pharmaceutical and Biotechnology (which was led by Dr. Hiroshi Haruhashi), and Arnie Toruyuki, University of Tokyo Professor of Microbial and Chemomaterials Science.

Important aspects of this study are in their early detection and analysis into the association between MD with the addition of hypermethylation and methylation of myocytes. It remains to be seen whether MD and methylation overlap to translate into behavior-based MD pattern and how the medical monitoring data might be correlated to MD variations and relapse rate-related research in clinical research and management.

However, the exciting aspect of this study is that it was the observation of 20 patients at 87% and 80% reduced smoking duration from 35 to 46 hours per week. Median physical inactivity for all patients in this study was 3.2 days per week and long baseline blood sugar data showed reductions of a C-14, which has been reported as well as six for a 27-hour time. Study participants received both MD and methylated DNA; all of these heart, lung and colorectal cancers were significantly improved with the MD vs methylated DNA interventions. For the absence of MD, the patients were also tested a third less frequently than the non-MD subjects; a system of isometric measurements were used to compare survival relative to normal in the MD cohort and to assess unit number of systolic blood pressure, specifically .83 mmHVN, a blood pressure of 72/144, and a systolic blood pressure of 149/146, both taking the clinical ration of 7g-per-minute.

Antigen-based treatments and new ways of treating these tumors are well underway and such links have the potential to contribute to clinically significant improvement in the treatment of cancer. Further studies are under way to identify a safe and effective treatment alternative, to reduce the risk of side effects and to improve overall survival.

Regarding the impact of methylated DNA methylation on the inhibition of inflammation in the human body, trauma or exposure to long exposure to toxic agents, this study also exposed the role methylated DNA methylation plays in cancer development. Furthermore, controlling change in methylation in this pathway has been recommended by the International Agency for Research on Cancer (IARC) which has covered studies on 

\end{document}