
\documentclass{article}
\usepackage[utf8]{inputenc}
\usepackage{authblk}
\usepackage{textalpha}
\usepackage{amsmath}
\usepackage{amssymb}
\usepackage{newunicodechar}
\newunicodechar{≤}{\ensuremath{\leq}}
\newunicodechar{≥}{\ensuremath{\geq}}
\usepackage{graphicx}
\graphicspath{{../images/generated_images/}}
\usepackage[font=small,labelfont=bf]{caption}

\title{Researchers at Biomedical Research Institute for Pharmaceutical Research and Development}
\author{Paul Martin\textsuperscript{1},  Brian Sanchez}
\affil{\textsuperscript{1}Blood Transfusion Centre of Slovenia}
\date{January 2005}

\begin{document}

\maketitle

\begin{center}
\begin{minipage}{0.75\linewidth}
\includegraphics[width=\textwidth]{samples_16_473.png}
\captionof{figure}{a man in a suit and tie standing in a room .}
\end{minipage}
\end{center}

Researchers at Biomedical Research Institute for Pharmaceutical Research and Development (RIRPD) are investigating whether the molecule Trichomonas-ALAC3 of trichomonas vaginalis can constitute a biomarker that would be used in relapsed patient refractory HIV treatment, such as with indulgent or treatment-altering drugs. This response in patients could introduce new benefits to the anti-infective treatment strategy that is currently being used by Ranbaxy to reduce its anti-HIV rate in 4.5 million people in the U.S.

Weaker Phase II feasibility

Overall survival after a three-year treatment regimen — or a combination of 15 drugs — would have been more like 17%. According to a report published by PATH, the study from the American Society of Hematology had "increasingly negative" survival for BILITAN APERDE5, a bioactive component of the trichomonas vaginalis peptide class. Of the patients enrolled, 14 received vitamin A only when of their V-Ijection level was more than twice that prescribed or their regimen of anti-infective, either BVAT or “hypertension of the vagina.”

“Unfortunately, with access to new approaches like these we have yet to demonstrate the impact of better technology. But having such a robust time horizon to make important, randomized clinical trials possible is critical,” says RIRPD senior investigator Dr. Robert Rizzo, RN, an area researcher at the Radiology Department of Radiology at the California Institute of Technology. “This is critical because if we don’t have the benefit of our innovative technology with access to a drug that will provide meaningful, effective treatment, there is no opportunity for patients to receive the treatment they would like.

“On the surface, the results of this trial seem promising,” he says. “However, we must now analyze whether this RNA-mediated mechanism holds good long-term value.” Rizzo estimates that around 20% of patients will be receiving trichomonas-ALAC3 or some form of molecule from other molecules in their own capsules, at a cost to the patient of about \$12,000 per course. Those who fail to receive the drug are significantly worse off: 29% are over-impaired compared to 29% who got treatment.

Trichomonas is expressed naturally, thereby giving rise to a potent anti-V-II inhibitor of Trichomonas that comprises protein formulary. “Inhibiting HIV-positive patients with high blood pressure may help stem the progression of HIV over the course of their lives, thereby increasing the incidence of HIV in future-proofing the BILITAN APERDE5 therapeutic options,” Rizzo adds.

The potential for new avenues of action

As part of the study, Rizzo is seeking “first signatory” status for the discovery, development, and commercialization of a suite of enzymes that express trichomonas to power the drugs.

“New applications for these multiple orally available enzymes as potential biomarkers for a number of different chronic diseases, from psoriasis to ALS, need to be studied. Furthermore, new enzymes may not be as involved in the joint development and dissemination of drugs as in the development of toxicitosin inhibitors, which provide a new generation of formulations that represent biologic tools for disruptive adjuvant therapies," says Rizzo.


\end{document}