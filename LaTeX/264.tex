
\documentclass{article}
\usepackage[utf8]{inputenc}
\usepackage{authblk}
\usepackage{textalpha}
\usepackage{amsmath}
\usepackage{amssymb}
\usepackage{newunicodechar}
\newunicodechar{≤}{\ensuremath{\leq}}
\newunicodechar{≥}{\ensuremath{\geq}}
\usepackage{graphicx}
\graphicspath{{../images/generated_images/}}
\usepackage[font=small,labelfont=bf]{caption}

\title{Based on early findings from an earlier part of the}
\author{Brandon Wu\textsuperscript{1},  Kimberly Rogers}
\affil{\textsuperscript{1}INFN - Istituto Nazionale di Fisica Nucleare}
\date{January 2014}

\begin{document}

\maketitle

\begin{center}
\begin{minipage}{0.75\linewidth}
\includegraphics[width=\textwidth]{samples_16_50.png}
\captionof{figure}{a young boy wearing a tie and glasses .}
\end{minipage}
\end{center}

Based on early findings from an earlier part of the clinical trial, physicians seem to have misguide the origins of a new family of proteins for administering RNA for relapsing multiple sclerosis (MS) treatment.

These new reports from scientists from the University of Rochester MD Anderson Cancer Center (URMC) and the University of Cincinnati MD Anderson Cancer Center (UCofacev), highlight the importance of molecular testing as a mechanism for RNA as a platform for compensating for slow-removal of disease-causing proteins, and generally good health for the body.

The genetic testing for molecular targeting of viral infections is also being studied as well, but it is most known for its risk factor for these multiple sclerosis therapies, as indicated by a portion of this latest report.

“Groteauviation is at least a hypothesis. However, we have been unable to investigate whether this assay is a feasible assay for any other genetic testing in MS,” said David Mulholland, Ph.D., associate professor of medicine and special projects manager at URMC.

In the Mitic line of reengineering for fatal disease, monoclonal antibodies are increasingly tested in a molecular testing process called sequencing. Biomarkers are currently being tested using sporadic marker protein profile (RORD) that is similar to proteins that are also expressed on cells. Pathogen recognition, code-based chemistry, and more multidisciplinary testing are used to identify these diseases. The programs were started in 2000 in the fields of immunology, pediatric medicine, cytology, and obstetrics, such as genetics and maternal lactation. When this research was done, there was a 37% increase in the number of adverse events in the PAL population, compared to other disease types.

However, this new potential uses of gene sequencing, directed at specifically specific protein sets, are not yet being studied in human trials. An initial trial may include the routine use of mutated genes to determine if they represent human immunodeficiency virus (AIDS). This may thus not be a next step forward in the identification of possible viral immune deficiencies. But this means that identifying the mutated or unrelated DNA sets in these current and potential upcoming studies would be a huge advantage.

“These studies also hold great promise in discovering new therapeutic approaches for disease by reversing blood bleeding, for example, when a compound derived from an HIV virus actually exploits immune cells causing the bleeding to stop,” said William Delore of the Medical School of the University of Toronto, who was not involved in the new study.

The new findings may eventually inform molecular diagnostic testing to detect infectious agents that may serve as markers of the disease burden, including non-fatal illness, severe arthritis, high blood pressure, and asthma.

The research was funded by the National Institutes of Health (NIH) Investigator's Fund, UnitedHealthcare Pathways (Northern Indiana) and \$1,000 grant awarded by the Broad Institute of MIT in partnership with the William S. and Mary M.S. Anderson Cancer Center and Rice University. Researchers recruited 15 patients enrolled in the Phase 2 clinical trial, which was co-sponsored by MED Global/the URMC emergency department and URMC chemo site. The patients were found to have at least some of the following characteristics:

more blood bleeding and body mass than normal

higher body mass than normal

higher body antibodies than normal

sensitivity to viral antigens

higher body antibodies than normal

higher body antibodies than normal

higher body antibodies than normal

lower body antibodies than normal

lower body antibodies than normal

Double negative and double negative antibodies were measured to determine an association between antibody levels and health-threatening conditions. Laboratory testers initially measured these amounts to confirm two recent genetic sequences, but these levels were eclipsed by adverse events. A subset of these events were not as severe, with patients lacking several types of added antibodies, such as antibodies from the TIA substance or neoplasms, as the findings outlined in the Mitic press release.

Original article on December 6, 2012. This story was originally published in the New York Times.


\end{document}