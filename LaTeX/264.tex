
\documentclass{article}
\usepackage[utf8]{inputenc}
\usepackage{authblk}
\usepackage{textalpha}
\usepackage{amsmath}
\usepackage{amssymb}
\usepackage{newunicodechar}
\newunicodechar{≤}{\ensuremath{\leq}}
\newunicodechar{≥}{\ensuremath{\geq}}
\usepackage{graphicx}
\graphicspath{{../images/generated_images/}}
\usepackage[font=small,labelfont=bf]{caption}

\title{Competition is heating up for protein studies and medications by}
\author{Benjamin Black\textsuperscript{1},  Johnathan Underwood,  Dawn Mcfarland,  Rebecca Craig,  Diana Hooper,  Anthony Kelley,  Troy Austin,  Teresa Orr,  Pedro Flowers}
\affil{\textsuperscript{1}Medical School of Southeast University}
\date{January 2013}

\begin{document}

\maketitle

\begin{center}
\begin{minipage}{0.75\linewidth}
\includegraphics[width=\textwidth]{samples_16_264.png}
\captionof{figure}{a couple of women standing next to each other .}
\end{minipage}
\end{center}

Competition is heating up for protein studies and medications by promoting a narrower view of treatment of patients with multiple myeloma (MMA) after several expansions of type 2D protein family. Major research will be conducted in the near future on the role of expanded and rapidly proliferating genes in MMA with promise to alter the course of this rare disease. The unmet medical need is a major feature in the development of revolutionary new treatments and clinical trials can significantly limit the proliferation of these modified genes.

In the first phase of studies at the Max Planck Institute for Human Genome Research in the Netherlands, the objective of this study is to formulate a classification of whether human and/or Genome recombinant genes A and B contain protein types A and B, particularly to see if they increase the ability of humans to participate in MMA. While the authors of the study are researching this question, the other authors in this report, thus a different focus, have also begun a new phase of research that is aimed at determining if the genetically tailored mutations do not take place in people with mMA.

It has been observed that mouse genetically modified human genetic mutations can be distinguished from those being treated with normal/non-mMA mutated ones. The UCL paper, published in Nature Genetics, cites research that has conducted on mice recently showing that the modified mutant of course does not have active stem cell lines. The authors suspect that these stem cell lines are implicated in the growth of triple fracture cell lines in human arm fractures.

The researchers took a number of significant human and several animal models into sequence which were determined to affect mutations including human and/or Genome recombinant genes A and B.

Co-authors of the study include Erin Wood, et al.; Jules Gilder, et al.; Ben Perkins, et al.; Saad Ali (Drs.), and Ramaj Shah (Physician), both of the Co-Proceedings; and Amiri Yayutan, Drs., and the Independent Clinical Diagnostic Laboratory at the site of Genome Hybrid International in Amsterdam.

After analysing the published examples, the investigators conclude that mammalian and genetically modified human scientists have identified the additional factors that may contribute to the difference between the modified and non-modified gene sequences from mouse and human tissues that were previously overlooked.

The paper says that other steps in this sequence will be discussed by other institutions such as Pinnacle of Proceedings, the Lombard Cell Institute in Chicago and Human Genome Research, Inc.


\end{document}