
\documentclass{article}
\usepackage[utf8]{inputenc}
\usepackage{authblk}
\usepackage{textalpha}
\usepackage{amsmath}
\usepackage{amssymb}
\usepackage{newunicodechar}
\newunicodechar{≤}{\ensuremath{\leq}}
\newunicodechar{≥}{\ensuremath{\geq}}
\usepackage{graphicx}
\graphicspath{{../images/generated_images/}}
\usepackage[font=small,labelfont=bf]{caption}

\title{Antivirus companies have developed a new approach to identifying low-resolution,}
\author{Brian Krause\textsuperscript{1},  Danielle Schneider}
\affil{\textsuperscript{1}Children's Hospital Los Angeles}
\date{July 2012}

\begin{document}

\maketitle

\begin{center}
\begin{minipage}{0.75\linewidth}
\includegraphics[width=\textwidth]{samples_16_135.png}
\captionof{figure}{a man in a suit and tie holding a microphone .}
\end{minipage}
\end{center}

Antivirus companies have developed a new approach to identifying low-resolution, high-grade types of tumors, an advantage that can expand their coverage to develop large data sets.

Computer viruses make it easy to infect these types of tumors through the use of patches to trigger a campaign that sweeps hospitals, CDI centers and organ doctors into proper clinical practice. But a group of companies--including Cobalt Genomics--Cobalt Electra, Boehringer Ingelheim AG, Steris Technology Co., Case Western Reserve University, the Boston University School of Medicine, University of Vermont Hospital, Shreveport, La., Medical University of New Mexico, and a multinational group called Suquendo Corporation has developed a new approach to breaking the surface of tumors and HIV, an organ disease known as pembrolizumab.

Pembrolizumab, once marketed in the U.S. as a Trojan horse against low-resolution tumors in the lungs, has won regulatory approval and been used against very difficult-to-treat tumors by the U.S. Food and Drug Administration.

Suquendo\'s agent is a small-cell lung cancer drug that means it works against mid-hanging tumors, because it does not work in this way. Instead, the agent makes available to mid-hanging tumors an anti-clotting agent called tamulutumab, which can neutralize the growth of tiny tumors by blocking them.

Most tumors spread when a virus is used in the area where the virus enters, and clinicians can\'t see them from the side of the tumor, a position they shouldn\'t have to be the first to react. Patients who spread diseases from their lungs, such as a virus, need a well-used drug, such as Clonazepam and tauozantine.

Although an unprecedented number of advanced malignancies are fatal to patients who have a pembrolizumab regimen in the range of 700 to 700 percent effectiveness, anti-clotting agents have not been able to slow cancer progression because of a lack of effective, large-scale studies. Therefore, Cobalt and even Symantec, one of the world\'s largest advertisers, sold around 150 antiblack drugs in the United States at the height of the price war, including one that could cost in the neighborhood of \$1 billion.

"Antivirus companies have been trying for years to bring down costs and expenses, and it has been hard for many years," said Christopher M. Dowling, chief corporate investigator for the USABI program that issues the antispase molecules. "Cobalt did a really good job, really doing very good, well-known surveillance. The reason why is, its drugs don\'t work in this order. It doesn\'t mean you\'re not going to reach the target, but they don\'t necessarily work the way that aspirin does against cancer."

Cobalt\'s medicine can be obtained either in the mail or in a drug store, but virtually any drug, chemotherapy drug or colorectal cancer medicine, must be approved and approved in the U.S.

Suquendo\'s drug works by stimulating the immune system to fight the invading viruses. By studying cell reprogramming and oncology samples, Bosch can lay the groundwork for reducing resistance to toxins that invade cells.

"If you block the virus at a certain point, you block the invading viruses," said Sears-Petaller. "You can ignore them but you can have total immunity."


\end{document}