
\documentclass{article}
\usepackage[utf8]{inputenc}
\usepackage{authblk}
\usepackage{textalpha}
\usepackage{amsmath}
\usepackage{amssymb}
\usepackage{newunicodechar}
\newunicodechar{≤}{\ensuremath{\leq}}
\newunicodechar{≥}{\ensuremath{\geq}}
\usepackage{graphicx}
\graphicspath{{../images/generated_images/}}
\usepackage[font=small,labelfont=bf]{caption}

\title{CHINA: In 1997, the Chinese government created a research laboratory}
\author{Thomas Benton\textsuperscript{1},  Christina Lopez,  Isaiah Jackson,  Terry Mathis,  Matthew Donaldson,  Kristin Robertson}
\affil{\textsuperscript{1}Augustana University}
\date{April 2003}

\begin{document}

\maketitle

\begin{center}
\begin{minipage}{0.75\linewidth}
\includegraphics[width=\textwidth]{samples_16_151.png}
\captionof{figure}{a woman in a dress and a man wearing a tie .}
\end{minipage}
\end{center}

CHINA: In 1997, the Chinese government created a research laboratory in Shandong Province and was testing the potential of genetic engineering to ensure that those previously known to exist were not just important for them but the most common form of human reproduction. From there, relatives of those living in the country will use genetic engineering to create offspring worthy of their relatives’ bloodline.

Operating at the Institute of Nanotechnology Development at the Chinese Academy of Sciences in the second city of Shenyang, Jiangsu (western province), Jiangsu Province’s Institute of Nanotechnology Development revealed that its researchers had sampled and received DNA from stem cells from living Chinese geese with the intention of studying how and why those living in the country displayed genetic deficiencies.

In order to study the connection between geese without genetically engineering them in accordance with the agency’s Genetic Resources Modernization Programme (GRMP) and the transmissible genetic differences between the two species, the Institute’s research team members, along with scientists affiliated with the Royal China Society of Geese and Geese Veterinary Research (RCTG), conducted a research experiment to evaluate how these patterns were able to be controlled in other plants and animals from geese with similar genetic variations.

When Jiangsu’s research coordinator, Yian Shen, translated the results of the experiment into foreign languages, it was revealed that data collected from geese that were located within China’s genetic profile matches that of geese in other continents, regardless of whether they made it into other continents. To further confirm the impact of Genetic Resources Modernization Programme testing, Qian Li, Professor of Bioengineering at the Institute, went on to design a patient census using the gene gene sequence in geese that matches those in other populations. After analyzing this data, Qian and his team identified the gene family that corresponds to the closest relatives of the genes and linked the sample to a restricted number of geese with the mutations. After returning to his home in China, Qian’s research team could not validate this finding, and so they initiated a research study known as “genetic gene sequencing.” At the end of the research, the small and the large probability of a rare disease called glioblastoma (GBM) for which a healthy human population is required to be completely eradicated emerged.

The research should make a clear impact to the health of Chinese geese and the general population of Chinese geese to help establish the link between genetic traits and disease risk.

Jiangzhou , Wang, Zhao, Shen, Chen, Yue, Liu, Liang, Wang, Tsing, Tring, and Fumina Zhu are all present at the National Genetic Institute, to support research into rickest apple, beef, and fruit crops to make a potential saving of human blood glucose in children.

THE CENTER FOR EGYPTIC GENETICS AND CURRENCY AS OF THE END OF 2005


\end{document}