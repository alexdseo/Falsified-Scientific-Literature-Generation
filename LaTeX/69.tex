
\documentclass{article}
\usepackage[utf8]{inputenc}
\usepackage{authblk}
\usepackage{textalpha}
\usepackage{amsmath}
\usepackage{amssymb}
\usepackage{newunicodechar}
\newunicodechar{≤}{\ensuremath{\leq}}
\newunicodechar{≥}{\ensuremath{\geq}}
\usepackage{graphicx}
\graphicspath{{../images/generated_images/}}
\usepackage[font=small,labelfont=bf]{caption}

\title{Infectious Disease researchers have discovered that AD8 agents may have,}
\author{Thomas Rodriguez\textsuperscript{1},  Erik Thompson,  Mikayla Kelly DDS}
\affil{\textsuperscript{1}University of Glasgow}
\date{April 2014}

\begin{document}

\maketitle

\begin{center}
\begin{minipage}{0.75\linewidth}
\includegraphics[width=\textwidth]{samples_16_69.png}
\captionof{figure}{a woman and a man pose for a picture .}
\end{minipage}
\end{center}

Infectious Disease researchers have discovered that AD8 agents may have, for decades, been co-infected with other pathogens such as Escherichia coli. For the first time in history, the immunocompromised mice have been part of a list of 36 living human participants who have died from lung infections associated with AD8 agents in the lungs of the elderly. The antibodies are believed to be derived from the results of their current multi-item study in E-lot Longitudinal Lung Immunogenicity and Function in Acute Lung Leak Diseases (LPLAC) at the Bowdoin Institute in West Germany. The findings will be published in the journal Cell Tuesday.

Recently, the Centers for Disease Control (CDC) has concluded that AD8 agents have no link to lung infections, but reports indicating that 99.3% of the lung infections being treated at CDC are related to the AD8 agents. The CDC indicates that the incidence of lung infections is growing faster than the incidence of lung cancers.

The study was carried out with a mouse population known as polarised death for 6,235 cases in the age group of 25-years-old from 1908-2010. As part of the study, 97.7% of polarised death for AD8 agents in the lungs was treated with AD8 anti-VEGF, an anti-TNF blocker. As part of the study, 213.3% of the AD8 antigens in the lungs were acquired from oral GDs. During the procedure, the antigens were injected into the lungs of the mice.

Dr. Sundanth Sookandja, lead author of the study and assistant professor at the UWM School of Public Health in King’s College London, London (UK), and co-authors of the lead paper say “we were surprised by the preliminary quality of the sample”, and went further to look at the link between AD8 and lung infection.

“The six antigens on blood samples were all from the Canadian Cipriani Lung Association. X03-G would have had four other antigens on blood samples, while one is a cancer-causing polybrominated diphenyl ether (PDEB).” In lung infection, there are two types of antibodies.

Professor Sookandja says the antibody at its most common form is ten-membrane. PMB is high GI that can cause pulmonary disruptions and other vascular issues that can lead to blood infection. Prof. Sookandja says “there is a risk of transmission of pulmonary disorders to other organs that can be managed with a systemic approach. This antibody belongs to the DTF-GN – DTF-GN – and is effective against AD8 in specific human diseases.”

PPEB was found to be common in males. Prof. Sookandja says “PPEB in mice has an average age of approximately 23 years. Our sample was also able to find a molecular mechanism within the cells for the AD8 specifically because the antibodies were injected into the lungs of the mice that were infected with a large number of the AD8 pathogens.”

Dr. Sookandja, who is on international interest in the AD8 research, says “We were more than satisfied with the detection of DTF-GN in humans in our mouse datasets and observed an increase in infection-related human lung infections in first-line lung cancer patients in research laboratories.”

Prof. Sookandja continued, “This study has implications for the fight against AD8-infected and/or flaviviruses (especially ASO-SMVs). We must keep up the pressure on the TSMC (T.C.J.) and ISR to increase immun immunization rates that can be achieved against AD8 with greater specificity and double-dipping.”

Prof. Sookandja says there is a story which the immunosuppressive potential of AD8 is “increasing by the day. There are not huge other medications available with AD8 which could ease lung infections, such as NSAIDs and TREs. We are not completely sure how many stomach drug addicts can benefit from AD8 but if it is high potency, it could be worth marketing to the public.

“AD8 can be made into anti-infected human immunodeficiency virus, which is a problem in our city, London. This research suggests there is not a widespread NOD

\end{document}