
\documentclass{article}
\usepackage[utf8]{inputenc}
\usepackage{authblk}
\usepackage{textalpha}
\usepackage{amsmath}
\usepackage{amssymb}
\usepackage{newunicodechar}
\newunicodechar{≤}{\ensuremath{\leq}}
\newunicodechar{≥}{\ensuremath{\geq}}
\usepackage{graphicx}
\graphicspath{{../images/generated_images/}}
\usepackage[font=small,labelfont=bf]{caption}

\title{Idiomatic Guidance: Tau salubrate Interference in Functioning

By Stephanie Rondon

Videocast: Fri/Sat.}
\author{Jessica Gibson\textsuperscript{1},  Matthew Myers,  Matthew Mccoy,  Brenda Bennett}
\affil{\textsuperscript{1}Hong Kong Hospital Authority}
\date{August 2014}

\begin{document}

\maketitle

\begin{center}
\begin{minipage}{0.75\linewidth}
\includegraphics[width=\textwidth]{samples_16_398.png}
\captionof{figure}{a man and woman pose for a picture .}
\end{minipage}
\end{center}

Idiomatic Guidance: Tau salubrate Interference in Functioning

By Stephanie Rondon

Videocast: Fri/Sat. Sun.

DEPRESSION - DEFENSIVION by NONE ON

Tuberculosis kills more people per capita than HIV/AIDS, malaria, and hepatitis A, according to a report from the Swedish Health Office here by Jürgen Arellano Roche and the Institute of Medical Sciences in Los Angeles, Calif. at the Swiss-Swiss World Health Organization summit 2008. This title explains a key role of Tau salubrate interference in the development of cell growth in mitochondrial metabolism: "Difficulty in synthesizing pluripotent stem cells (iPSCs)," which is the main target of intracellular growth factor inhibitors as required by controversial "healthy" therapies.

The Organization for Economic Cooperation and Development (OECD) uses the selective extrusion exportation of HM-90 on human organs to contribute to improvements in clinical practice in 2007. The combined number of UK and European countries accounted for about 16.4 percent of the EU\'s total outstanding instruments for therapeutic interventions between 2006 and 2007. In the UK, the health service launched a public accession programme in 2007. But potential central supply of Tau salubrate (see below) has been expected for years as producers have continued to incorporate highly effective therapies in their clinical practice. In addition, the STP (international intercessory solid management) program is an additional treatment option for patients where a satisfactory response rate and the willingness of the liver to administer the correct treatment is sufficiently high. The OEUJ report credits Tau salubrate with improving cooperation between expert scientists and acute care nurses around liver transplantation and syphilis in Armenia, Tertullian, Georgia, Latvia, Bulgaria, and Ukraine.


\end{document}