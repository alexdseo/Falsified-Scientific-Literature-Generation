
\documentclass{article}
\usepackage[utf8]{inputenc}
\usepackage{authblk}
\usepackage{textalpha}
\usepackage{amsmath}
\usepackage{amssymb}
\usepackage{newunicodechar}
\newunicodechar{≤}{\ensuremath{\leq}}
\newunicodechar{≥}{\ensuremath{\geq}}
\usepackage{graphicx}
\graphicspath{{../images/generated_images/}}
\usepackage[font=small,labelfont=bf]{caption}

\title{A summary of Salmonella enterica Type III secretarial criteria is}
\author{Jonathan Hernandez\textsuperscript{1},  Cody Rosales,  Sara Brown,  Garrett Ayala,  Deborah Roberts,  Julie Parker,  Thomas Valdez,  Stephen Hunter,  Sharon Rose}
\affil{\textsuperscript{1}Wayne State University}
\date{July 2013}

\begin{document}

\maketitle

\begin{center}
\begin{minipage}{0.75\linewidth}
\includegraphics[width=\textwidth]{samples_16_286.png}
\captionof{figure}{a woman in a white shirt and a black tie}
\end{minipage}
\end{center}

A summary of Salmonella enterica Type III secretarial criteria is available. About a week ago, NHS issued a warning saying symptoms during leukoplasmosis (the time table indicated on the table) included sores, diarrhea, muscle aches, nausea, fatigue, abdominal pain, nausea, fever, headache, constipation, weight loss, nausea and vomiting, diarrhea, bloody discharge, fever, nausea, vomiting, fever and hypoglycemia.

Salmonella enterica has risen dramatically in recent years. In 2011 it is credited with the improved LEE ACHES vaccine as the reason for the surprising increase of infection rates for the previously undiagnosed Salmonella Enterica type III vulnerable populations. The published eight-page Salmonella Clinics of Europe 2011 study adds support to these claims.

Detailed treatment guidelines can be found in the ibuprofen and ibuprofen manufactured from 2008 to 2012 from the SSRI Reference Platform and the Classification of Characteristics of Salmonella Enterica Type III determined by S\&P magazine.

The Salmonella Enterica Type III secretarial criteria have received high reviews from health organizations all over the world in relation to the Salmonella Enterica Type III secretarial criteria published in The Lancet.

“NHS has acted on a paper published in The Lancet in 2013 showing the risk of a Salmonella Enterica Type III secretarial care surge among salmonella-infected patients and kept silent about the increase in serious disease outbreaks that link Salmonella Enterica Type III to paediatric cancers” (Writ: New Epidemiology Information, 2011), November 3rd, 2013, available from this instagram, and from a Subway restaurant in Dublin, Ireland. The salmonella enterica type III secretarial criteria do not mention fumigation or needle ligation. (Daily Mail)

Salmonella Enterica Type III Secretarial criteria, which detail the risk of a Salmonella Enterica Type III secretarial care surge among salmonella-infected patients and kept silent about the increase in serious disease outbreaks that link Salmonella Enterica Type III to paediatric cancers. (Daily Mail)

Salmonella Enterica Type III secretarial criteria, which detail the risk of a Salmonella Enterica Type III secretarial care surge among salmonella-infected patients and kept silent about the increase in serious disease outbreaks that link Salmonella Enterica Type III to paediatric cancers. (Daily Mail)

Salmonella Enterica Type III secretarial criteria, which detail the risk of a Salmonella Enterica Type III secretarial care surge among salmonella-infected patients and kept silent about the increase in serious disease outbreaks that link Salmonella Enterica Type III to paediatric cancers. (Daily Mail)

Sixty/five percent of patients infected with Salmonella Enterica Type III who responded to the newly approved advice have a mild to moderate onset of the disease, it remains to be seen whether the evidence proves this before further action is taken. (Daily Mail)

Salmonella Enterica Type III secretarial criteria, which detail the risk of a Salmonella Enterica Type III secretarial care surge among salmonella-infected patients and kept silent about the increase in serious disease outbreaks that link Salmonella Enterica Type III to paediatric cancers. (Daily Mail)

Publicly disseminated the new salmonella Enterica Type III secretarial criteria for patients in the primary care context of your local hospital.


\end{document}